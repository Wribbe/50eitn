\documentclass[10pt]{article}
\usepackage{enumitem}
\usepackage{parskip}
\usepackage[T1]{fontenc}
\usepackage[utf8]{inputenc}
\usepackage{listings}
\usepackage{hyperref}

\newcommand{\command}[1]{\texttt{#1}}

\begin{document}

  % Title page.
  %------------

  \thispagestyle{empty}
  \vspace*{3cm}
  \begin{center}
    \huge{EITN50 -- Advanced Computer Security} \\
    \vspace{0.3cm}
    \LARGE{Anatomy of an Exploit} \\
    \vspace{1cm}
    \large{by: \\ \vspace{0.2cm}
        Stefan Eng \texttt{<atn08sen@student.lu.se>} \\
        Rasmus Olofzon \texttt{<muh11rol@student.lu.se>}
        } \\
  \end{center}

  % First page
  %-----------

  \newpage

  \section*{Introduction}

  \section{Assignments}

    \subsection{Assignment 1 -- Running the exploit}

      Executing the exploit.py file without the software started resulted in a
      command prompt displays the following:
      \begin{verbatim}
        [*] Connecting to 192.168.0.1...
      \end{verbatim}
      And nothing else happens.

      Analyzing the exploit script reveals that it tries to connect to and
      exploit an instance of Easy File Management Web Server v5.3 to trigger a
      remote buffer overflow. When the exploit is successful, it should close the
      management software and bring up an instance of the windows calculator.

      After starting the web management software and running the exploit again,
      this is what happens.

    \subsection{Assignment 2}

      Descriptions of the different gadgets present in the payload code:

      \begin{description}
        \item[junk0]{placeholder}
        \item[call\_edgx]{placeholder}
        \item[junk1]{placeholder}
        \item[ppr]{placeholder}
        \item[crafted\_jmp\_esp]{placeholder}
        \item[test\_bl]{placeholder}
        \item[kungfu]{placeholder}
        \item[nopsled]{placeholder}
        \item[shellcode]{placeholder}
      \end{description}

      This is a schematic representation of the stack before the payload has
      been deployed.

      \begin{verbatim}
        [schematic repreentation of stack before payload]
      \end{verbatim}

      And this is a schematic representation of the stack after the payload
      has been deployed.

      \begin{verbatim}
        [schematic repreentation of stack after payload]
      \end{verbatim}

    \subsection{Assignment 3}

      After attaching to the software and putting a breakpoint at the address
      0x00468702, the following instructions were executed until the executable
      hit the shell-code payload.

      \begin{verbatim}
        [list of all the instructions with NOP sections compressed]
      \end{verbatim}

      Analyzing the instructions the following were deemed most vital for the
      exploit to work.

      \begin{verbatim}
        [Numbered list of exploites picked from the list above]
      \end{verbatim}

      The purpose of each of these instructions is documented below.

      \begin{verbatim}
        [Enumerated / Description environment that explains all
        the highlighted instructions from above]
      \end{verbatim}


    \subsection{Assignment 4}

    \subsection{Assignment 5}

    \subsection{Assignment 6}

    \subsection{Assignment 7}

\end{document}
