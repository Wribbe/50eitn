\documentclass[10pt]{article}
\usepackage{enumitem}
\usepackage{parskip}
\usepackage[T1]{fontenc}
\usepackage[utf8]{inputenc}
\usepackage{listings}
\usepackage{hyperref}
%\usepackage{fullpage}

\newcommand{\command}[1]{\texttt{#1}}
\newcommand{\Q}[2]{\textit{\textbf{Q#1}: #2}}
\newcommand{\A}[1]{\textbf{A}: #1}

\begin{document}

  % Title page.
  %------------

  \thispagestyle{empty}
  \vspace*{3cm}
  \begin{center}
    \huge{EITN50 -- Advanced Computer Security} \\
    \vspace{0.3cm}
    \LARGE{TPM - Trusted Platform Module} \\
    \vspace{1cm}
    \large{by: atn08sen \&\& muh11rol} \\
  \end{center}

  % First page
  %-----------

  \newpage

  \section*{Introduction}

    The purpose of the TPM project is to familiarize the students with the
    TPM12 specification through the interaction with a simulated TPM. The more
    specific learning goals are:

    \begin{itemize}
      \item{Introduce TPM usage concept.}
      \item{Understand TPM1.2 key hierarchy.}
      \item{Understand binding, sealing and attestation.}
      \item{Understand key migration.}
      \item{Write a simple TPM application.}
    \end{itemize}

    This report is based on TPM-handout v1.0 (2017-09-17).

  % Set section counter to 2 to match assignment numbers.
  \setcounter{section}{2}

  \section{Assignments}

    \subsection{Setting up the TPM environment}

      Get the virtual machine up and running:

      \begin{enumerate}
        \item{Copy image archives from network drive S:\textbackslash Courses\textbackslash eit\textbackslash EITN50\textbackslash Project - TPM to
          local lab room computer}
        \item{Extract the images using 7-zip or similar.}
        \item{Add the TPM1, TMP2 and TSS images as machines to virtualbox:
          \command{Machine > Add}.}
        \item{Start the TPM1 and TSS virtual machines.}
        \item{Login on TPM1 with user: \emph{pi} and password: \emph{tpm}.}
        \item{Login on TSS with user: \emph{tss} and password: \emph{lab}.}
      \end{enumerate}

      Ensure that the necessary environment variables are set to correct values
      for TPM1:

      \begin{enumerate}
        \item{Open terminal on TPM1.}
        \item{Check IP address with: \command{ifconfig eth0}, should be 10.0.2.14.}
        \item{Check the environment variables of TPM1 with: \command{env | grep
          \textasciicircum TPM}.}
      \end{enumerate}

      The last command should return the following:
      \begin{quote}
        pi@TPM1 ~ \$ env | grep \textasciicircum TPM\\
        TPM\_PORT=6545\\
        TPM\_SERVER\_NAME=localhost\\
        TPM\_PATH=/home/pi/tpm/tpm4720/tpmstate\\
        TPM\_SERVER\_PORT=6545
      \end{quote}

      Ensure that the necessary environment variables are set to correct values
      for TSS:

      \begin{enumerate}
        \item{Open terminal on TSS.}
        \item{Check the environment variables of TSS with: \command{env | grep
          \textasciicircum TCSD}.}
      \end{enumerate}

      Which should return:

      \begin{quote}
        tss@TSS ~/Desktop/share \$ set | grep \textasciicircum TCSD\\
        TCSD\_TCP\_DEVICE\_HOSTNAME=10.0.2.15\\
        TCSD\_TCP\_DEVICE\_PORT=6545\\
        TCSD\_USE\_TCP\_DEVICE=true
      \end{quote}

      The variable TCSD\_TCP\_DEVICE\_HOSTNAME, needs to be changed from *.15 to
      *.14, which can be done by running:

      \command{export TCSD\_TCP\_DEVICE\_HOSTNAME=10.0.2.14}

      And the same goes for TPM\_SERVER\_HOSTNAME:

      \command{export TPM\_SERVER\_HOSTNAME=10.0.2.14}

      Start the emulator on the TPM1 machine:

      \command{tpm\_server}

      and connect to the running instance it on the TSS machine:

      \command{tpmbios} \\
      \command{sudo -E /usr/local/sbin/tcsd -e -f}

      open a second terminal on the TSS machine, this window will be used to
      issue all commands to the remote TPM emulator.

    \subsection{Generating EK, ownership and SRK public key}

      Use the command-window that is open on the TSS instance and run
      \command{createek}. This generates the EK public key and prints
      it to the terminal running the \command{tpm\_server} command
      on the TPM1 computer. \\

      Scrolling through the output yields the following:

      \begin{quote}
        ...\\
        TPM\_RSAGenerateKeyPair: length of n,p,q,d = 256/128/128/256\\
        TPM\_Key\_GenerateRSA: Public key n f7 f8 17 68\\
        TPM\_Key\_GenerateRSA: Exponent length 3\\
        01 00 01 \\
        TPM\_Key\_GenerateRSA: Private prime p fd d4 19 9d\\
        TPM\_Key\_GenerateRSA: Private prime q fa 17 28 ad\\
        TPM\_Key\_GenerateRSA: Private key d e4 c5 57 5a\\
        ...\\
        TPM\_SymmetricKeyData\_SetKeys: userKey f3 de c2 b9\\
        ...
      \end{quote}

      which gives:

      \begin{itemize}
        \item{Public key -- 9d c7 a2 12}
        \item{Private key -- 7d 80 b5 01}
      \end{itemize}

      The next step is to take ownership of the simulated TPM. This is done by
      running the following command in the TSS command window:

      \command{takeown -pwdo pass -pwds pass}

      where both the owner password and
      the storage root key password are \textit{pass}.

      After the ownership is established, the public SRK key can be dumped using
      the following command:

      \command{ownerreadinternalpub -hk 40000000 -of srk.pub -pwdo
      pass}

      Which dumps the public key to the file \textit{srk.pub} and prints:

      \begin{quote}
        \texttt{TPM\_IO\_Write: length 335\\
        ...\\
        \ \ \ CD ED 0E 0A 19 80 B9 5A A6 F8 \\
    1E 30 D2 4E 03 6E 28 55 48 94 2A 1A 15 66 C7 9B \\
    CD DE FD 01 38 FC EA 42 CE 2B CC 1D 15 0D 2F 73 \\
    83 0A 5E 35 D9 53 64 3D 7C E6 BF 6E 39 91 A7 B1 \\
    21 12 46 11 CC 35 D2 2B C2 3B FE 6E 98 C7 9A 28 \\
    DA 05 A0 B3 D2 AB 4A F5 DB 2F B0 55 63 93 62 3E \\
    0B 53 88 E4 44 CF 73 8C 05 CF 48 07 4B 15 B1 89 \\
    25 B8 FD 3D 07 C3 2A 81 64 6D C8 29 DD CF A9 05 \\
    C7 9E D9 01 50 96 7C 08 E0 5D 29 D0 26 49 D8 90 \\
    7C B7 C3 A2 14 2E F9 00 F8 C7 C0 8A DD EE 43 EC \\
    FB ED 08 4F D9 D4 85 1A AB 5D 75 1B D7 FC D9 FB \\
    72 BD F4 8D 55 C7 53 F9 1B 2F AE 84 A7 1A 4E 1A \\
    5B A5 39 7E D7 4F 92 5C ED 27 55 AC 9B 2D A5 2E \\
    40 44 0D CE BD 22 FE 22 B7 DE 3E 59 F1 40 E5 CA \\
    33 FB 0B EC 9D E9 09 99 C3 FA DF 1E 22 42 7B A4 \\
    5E D9 93 09 4C 8C E8 B6 6B F9 92 3E E5 4E 6C 91 \\
    7E 8B F9 FE 62 E1 \\
    ...
       }
      \end{quote}

      to the terminal running the TPM-emulator.
      and adding the \command{-v} flag to the \\
      \command{ownerreadinternalpub} command
      prints the same bytes to the TSS command window.

  \subsection{Key hierarchy}

    \Q{1}{Describe the difference between an identity and signature key.}\\
    \A{
      An identity key is an alias for the Endorsement Key. The alias is used
      to mitigate privacy and security issues. Without the alias is it easy to
      link a single EK to a single user. The alias also avoids reducing the
      entropy of the EK by not using it directly. An AIK (Attestation Identity
      Key) and signing keys are limited in which operations they can be used
      in. It's possible to use the AIK in the \command{TPM\_Quote} and
      \command{TPM\_CertifyKey}, but not in \command{TPM\_Sign}, where a
      signing key should be used. The general difference between identity and
      signature keys is that signing keys are used to sign arbitrary data and
      identity keys are used for remote attestation.
  }

    \Q{2}{Which keys can be used for file encryption?}\\
    \A{Storage keys.}

    \Q{3}{
      There is a key that exists, but using it is not recommended. Which key
      is that, and why does it exist?
    }\\
    \A{
      Legacy keys. These exist in TPM 1.2 only in order to preserve compatibility.
    }

    \subsubsection{The key hierarchy}

    The table produces the following structure:

\begin{lstlisting}
        SRK
        /  \
       H    A
          / | \
         B  F  G
       / | \
      C  D  E
\end{lstlisting}

    While it is possible to create the structure, any non-migratable key that
    is created as a child to a migratable key itself becomes migratable
    automatically.  A more correct structure would be to swap G and C.
    Then all the non-migratable keys are direct children of the non-migratable A.

    \subsubsection{Key hierarchy creation}

    H -- Identity key, parent: SRK

    \command{identity -la H -pwdk pass -pwds pass -v12 -ok H -pwdo pass -la idH -v}

    \command{loadkey -hp 40000000 -ik H.key -pwdp pass}

    output: New Key Handle = C4D5B9E3

    A -- Non migratable storage key, parent: SRK.

    \command{createkey -kt e -pwdk pass -pwdp pass -ok A -hp 40000000}

    \command{loadkey -hp 40000000 -ik A.key -pwdp pass}

    output: New Key Handle = D5701362

    B -- Migratable storage key, parent: A

    \command{createkey -kt em -pwdk pass -pwdp pass -pwdm pass -ok B -hp D5701362}

    \command{loadkey -hp D5701362 -ik B.key -pwdp pass}

    output: New Key Handle = 2101949B

    \textit{Note here that C is specified to be a non migratable key.
    However, a migratable key cannot have a non migratable child.
    Thus, C must be migratable.}

    C -- A migratable sign key, parent: B

    \command{createkey -v -kt sm -pwdk pass -pwdp pass -pwdm pass -ok C -hp 2101949B}

    \command{loadkey -hp 2101949B -ik C.key -pwdp pass}

    output: New Key Handle = 398116CC

    D -- A migratable sign key, parent: B

    \command{createkey -kt sm -pwdk pass -pwdp pass -pwdm pass -ok D -hp 2101949B}

    \command{loadkey -hp 2101949B -ik D.key -pwdp pass}

    output: New Key Handle = 45130233

    E -- A migratable bind key, parent B.

    \command{createkey -kt bm -pwdk pass -pwdp pass -pwdm pass -ok E -hp 2101949B}

    \command{loadkey -hp 2101949B -ik E.key -pwdp pass}

    output: New Key Handle = 6C66DD3A

    F -- A non migratable sign key, parent A.

    \command{createkey -kt s -pwdk pass -pwdp pass -ok F -hp D5701362}

    \command{loadkey -hp D5701362 -ik F.key -pwdp pass}

    output: New Key Handle = 8D3D52C8

    G -- A migratable sign key, parent A.

    \command{createkey -kt sm -pwdk pass -pwdp pass -pwdm pass -ok G -hp D5701362}

    \command{loadkey -hp D5701362 -ik G.key -pwdp pass}

    output: New Key Handle = DFE08D64

\section{Assignment 4: Key Migration}
\subsection{Questions}
\begin{enumerate}
    \item {Is it possible for a migratable key to be the parent of a non-migratable key?}
	\begin{itemize}
        \item {No. When migrating a key, all its children needs to follow
        (much like grafting of real trees, one cannot remove the middle part
        of a branch and leave the end hanging in the air).}
	\end{itemize}

    \item {Which command is the first to be executed when performing a key migration?}
	\begin{itemize}
	    \item {TPM\_AuthorizeMigrationKey  $\rightarrow$  TPM\_CreateMigrationBlob  (i. e., authorize migration key command)}
	\end{itemize}

    \item {Give a short description of the command TPM\_ConvertMigrationBlob.}
	\begin{itemize}
	    \item {This command enables a migration-blob to be loaded into a TPM as a 'normal', wrapped blob.\\
From TPM part 3 pdf: "Loading one of these wrapped blobs does not require authorization, since correct blobs were created by the TPM under Owner authorization, and unwrapped blobs cannot be used without Owner authorisation." \\
So: unwrapped blobs cannot be used without owner authorization, wrapped without. This command makes a blob usable for the "new" TPM, so that it can use the keys in the blob without prompting the user.}
	\end{itemize}

    \item {Which TPM command loads the migrated keys into the TPM?}
	\begin{itemize}
	    \item {TPM\_LoadKey2}
	\end{itemize}

    \item {Is it the TPM or the TSS that handles the transfer of the migration blob? }
	\begin{itemize}
	    \item {The TSS. Since the TPM only handles conversion of input and output
data, the transfer of the resulting/necessary data is handled by the software
stack (TSS).}
	\end{itemize}
\end{enumerate}

\subsection{Instructions: Key migration in the TPM emulator}
\textit{First: create a migration key blob on TPM1 for B that can be saved and then reloaded on TPM2.}

TPM2: moved existing TPM state (00.permall) to other directory. Checked IP address with \command{ifconfig eth0}, was 10.0.2.15. Ran \command{tpm\_server}

TSS: Made a new directory for TPM2 named TPM2. Ran following commands: \\
\begin{quote}
(in dir TPM1:) cd ../TPM2 \\
export TPM\_SERVER\_NAME=10.0.2.15 \\
export TCSD\_TCP\_DEVICE\_HOSTNAME=10.0.2.15 \\
tpmbios \\
(in other terminal window, TSS\#2) sudo -E /usr/local/sbin/tcsd -e -f \\

createek \\
takeown -pwdo pass -pwds pass\\

tpm2\_mig: SRK, migratable storage key \\
createkey -kt em -pwdp pass -pwdk pass -pwdm pass -ok tpm2\_mig -hp 40000000\\
loadkey -hp 40000000 -ik tpm2\_mig.key -pwdp pass\\
New Key Handle =  2E92949F\\

cp tpm2\_mig.key ../TPM1/tpm2\_mig.key \\
cd ../TPM1
\end{quote}

Shut down daemon in TSS\#2.

TPM1:  \command{tpm\_server } \\

TSS: \\
\begin{quote}
export TPM\_SERVER\_NAME=10.0.2.14 \\
export TCSD\_TCP\_DEVICE\_HOSTNAME=10.0.2.14 \\
tpmbios \\
(TSS\#2) sudo -E /usr/local/sbin/tcsd -e -f \\

(TSS\#1) export TPM\_SERVER\_NAME=10.0.2.14 \\
export TCSD\_TCP\_DEVICE\_HOSTNAME=10.0.2.14 \\

migrate -hp D5701362 -pwdp pass -pwdo pass -im tpm2\_mig.key -pwdk pass -pwdm pass -ok migBlob -ik B.key\\
Wrote migration blob and associated data to file.

cp migBlob ../TPM2/

\end{quote}
TPM2 (changed analogous to above):

\begin{quote}
loadmigrationblob -hp 2E92949F -if migBlob -pwdp pass \\
Successfully loaded key into TPM.\\
New Key Handle = BB2DD42A\\
\end{quote}

Load key C, D and E into TPM2:\\
TPM1:
\begin{quote}
migrate -hp 2101949B -pwdp pass -pwdo pass -im tpm2\_mig.key -pwdk pass -pwdm pass -ok migBlobC -ik C.key\\
Wrote migration blob and associated data to file.\\
cp migBlobC ../TPM2/

migrate -hp 2101949B -pwdp pass -pwdo pass -im tpm2\_mig.key -pwdk pass -pwdm pass -ok migBlobD -ik D.key\\
Wrote migration blob and associated data to file.\\
cp migBlobD ../TPM2/

migrate -hp 2101949B -pwdp pass -pwdo pass -im tpm2\_mig.key -pwdk pass -pwdm pass -ok migBlobE -ik E.key\\
Wrote migration blob and associated data to file.\\
cp migBlobE ../TPM2/

\end{quote}
TPM2:
\begin{quote}
loadmigrationblob -hp 2E92949F -if migBlobC -pwdp pass \\
Successfully loaded key into TPM.\\
New Key Handle = 65162BE4

loadmigrationblob -hp 2E92949F -if migBlobD -pwdp pass \\
Successfully loaded key into TPM.\\
New Key Handle = 044C3BF8

loadmigrationblob -hp 2E92949F -if migBlobE -pwdp pass \\
Successfully loaded key into TPM.\\
New Key Handle = 2B4C72AF
\end{quote}

As can be seen above, migrating keys C, D and E into TPM2 worked.
Since all three keys are migratable (C was changed to migratable
in order to be able to created, see above) it is as it should be.

\subsection{Questions}
\begin{enumerate}
    \item {There are other ways to migrate keys. When do you use a key of type TPM\_KEY\_USAGE = TPM\_Migrate? }
	\begin{itemize}
	    \item {"TPM\_KEY\_MIGRATE; 0x0016; This SHALL indicate a key in use for TPM\_MigrateKey" - from TPM commands document.}
	\end{itemize}
    \item {What is the rewrap option of the migrate command used for? }
	\begin{itemize}
        \item {To directly transfer a key to another TPM. The re-wrap flag tells the TPM
        to re-wrap (decrypt->encrypt) the key with a new parent, which enables that parent
        to load the key as a normal encrypted element}
	\end{itemize}
\end{enumerate}

%\begin{enumerate}
    %\item {}
	%\begin{itemize}
	    %\item {}
	%\end{itemize}
%\end{enumerate}

\section{Assignment 5: Extending values to PCRs}
\subsection{Questions}
\begin{enumerate}
    \item {Describe one TPM command that can be used to extend a SHA-1 digest to a PCR.}
    \begin{itemize}
        \item {\texttt{TPM\_SHA1CompleteExtend}. “This capability terminates a pending SHA-1 calculation
        and EXTENDS the result into a Platform Configuration Register using
        a SHA-1 hash process.” (TPM-part3 p.160).}
    \end{itemize}
    \item {Describe which TPM command that can be used to read a PCR value.}
    \begin{itemize}
        \item {\texttt{TPM\_PCRRead}. “The \texttt{TPM\_PCRRead} operation provides non cryptographic reporting
        of the contents of a named PCR.” (TPM-part3 p.162)}
    \end{itemize}
\end{enumerate}

\subsection{SHA-1 calculation and PCR extending using the TPM emulator}
Following commands and their printouts are the result of hashing the
\texttt{tpmbios} file and extending that result to PCR 11 of TPM2.
\begin{quote}
sha -if /home/tss/tpm/tpm4720/libtpm/utils/tpmbios -ix 11\\
SHA1 hash for file '/home/tss/tpm/tpm4720/libtpm/utils/tpmbios': \\
Hash: 55ac0462404445623f38fdae9adf87d487125874\\
New value of PCR: dba8c73876627a1e4439627b64c96c8f9c8d404a

pcrread -ix 11 -v \\
Current value of PCR 11: dba8c73876627a1e4439627b64c96c8f9c8d404a
\end{quote}

\section{Assignment 6: File encryption}
\subsection{Questions}
\begin{enumerate}
    \item {Why is TSS\_Bind a TSS command, and not a TPM command?}
    \begin{itemize}
        \item {Because binding is done outside of the TPM. Binding is encrypting
        data using the public key of a bind key and this is, as per
        asymmetric encryption principles, doable from anywhere.}
    \end{itemize}
    \item {Give some differences between Data binding and Data sealing.}
    \begin{itemize}
        \item {As stated in above answer, binding uses the public part of a key.
        Sealing uses binding: binds data with a non-migratable key, and embeds PCR
         values into
        the output. This can be used to in a later stage make certain a TPM is in
         the same state as when the
        data was sealed (by checking the values of the same PCRs that were used
        in the sealing).}
    \end{itemize}
    \item {Can a key used for data sealing be migrated to another TPM?}
    \begin{itemize}
        \item {No. The requirement on using a non-migratable key prevents this.}
    \end{itemize}
\end{enumerate}
\subsection{Data binding using the TPM emulator}
\begin{quote}
createkey -kt bm -ok bindMig -hp D5701362 -pwdp pass -pwdk pass -pwdm pass \\
touch textToBind.txt\\
listkeys -v >> textToBind.txt \\
bindfile -ik bindMig.pem -if textToBind.txt -of bindTextOutput \\
Error 'Error from OpenSSL library' from TSS\_Bind
\end{quote}

\subsection{Data sealing using the TPM emulator}
\section{Assignment 7: TPM Authentication}

  \subsection{Questions}

    Question: Could the verifyfile command be done by another TPM? \\
    Answer: Yes, it only uses the public part of the key to verify the signature.

    Question: Which TPM command is used to decrypt the file? \\
    Answer: TPM\_UnBind is used to decrypt files.

    Question: Can sealing be used in decryption based authentication instead of
    verifyfile? \\
    Answer: Not really, the PCR-values will not match from one TPM to another, so only
    the original TPM can do the unsealing.

  \subsection{Authentication exercises}

    Given two parties, P1 and P2, P1 can authenticate itself to P2 using either
    signing or encryption. Using signing, P2 can send a file that P1 signs with
    the private part of their asymmetric key-pair, and then P2 can use the
    public part to verify the signature of the file it sent.

    Using encryption, P2 can encrypt a message-file with the public part of the
    key belonging to P1, and then send the file to P1. If P1 then sends back
    the original unencrypted message,they prove that they possess the private
    part of the key-pair that enables decryption.

    \subsubsection{Authentication through signing}

      In this case TPM1 will authenticate itself to TPM2 through signing.
      First, TMP1 will need a signing-key.

\begin{verbatim}
createkey -hp 40000000 -pwdp sss -ok 3_7_2_sign_key -kt s
loadkey -hp 40000000 -ik 3_7_2_sign_key.key -pwdp sss
New Key Handle = E8CA7920
\end{verbatim}

      The key can then be used to sign a file, producing a signature.

\begin{verbatim}
signfile -hk E8CA7920 -if 3_7_3_file_sign.txt \
  -os 3_7_3_file_sign.signature
\end{verbatim}

      Then, given the original file, the *.signature file and the public part of
      the key used (the *.pem file), TPM2 can verify that the signature comes from
      TPM1 through the verifyfile command.

\begin{verbatim}
verifyfile -is 3_7_3_file_sign.signature -if 3_7_3_file_sign.txt \
  -ik 3_7_2_sign_key.pem
\end{verbatim}

      Which yields an error if the signature could not be verified.

    \subsubsection{Authentication through encryption}

      This process also needs a key, this time it's a binding key created by
      TPM1.

\begin{verbatim}
createkey -hp 40000000 -pwdp sss -ok 3_7_2_bind_key -kt b
loadkey -hp 40000000 -ik 3_7_2_bind_key.key -pwdp sss
New Key Handle = 64BCA797
\end{verbatim}

      Now the public part of the key can be shared with TPM2, which it uses it
      to bind a file.

\begin{verbatim}
cat 3_7_3_file_bound.txt
STRING IN BOUND FILE!
bindfile -ik 3_7_2_bind_key.pem -if 3_7_3_file_bound.txt \
  -of 3_7_3_file_bound.bound
\end{verbatim}

      Which produces the encrypted file \texttt{3\_7\_3\_file\_bound.bound}
      that is then made available to TPM1. TPM1 can then decrypt the file and
      return the contained message to TPM2 in order to prove that it is TPM1.

\begin{verbatim}
unbindfile -hk 64BCA797 -if 3_7_3_file_bound.bound \
  -of 3_7_3_file_bound.unbound
cat 3_7_3_file_bound.unbound
STRING IN BOUND FILE!
\end{verbatim}

\section{Assignment 8: Attestation}

  Attestation is a functionality that enables a platform to provide proof that
  the software it is running is correct and trusted. This proof, which can be
  based on signing or encryption can then be sent to an outside observer for
  verification.

  \subsection{Signature based attestation}

    A attestation identity key (AIK) is needed in order to preform any signing.
    Which is created with the following command:
\begin{verbatim}
identity -pwdo ooo -la assignment8 -pwds sss -pwdk kkk \
  -ok assignment_8_aik
\end{verbatim}

    Then the key needs to be loaded into the TPM:

\begin{verbatim}
loadkey -hp 40000000 -ik assignment_8_aik.key -pwdp sss
New Key Handle = 1F605F33
\end{verbatim}

    The digest of the binary file 'tmpbios' is then used to populate PCR
    register 11:

\begin{verbatim}
sha -if /home/tss/tpm/tpm4720/libtpm/utils/tpmbios -ix 11
SHA1 hash for file '/home/tss/tpm/tpm4720/libtpm/utils/tpmbios':
Hash: 55ac0462404445623f38fdae9adf87d487125874
New value of PCR: 21f2d0f812f2492bb376df8fdb62bbeeec86cd81
\end{verbatim}

    Which then can be used to run the quote command to verify the AIK.

\begin{verbatim}
/home/tss/tpm/tpm4720/libtpm/utils/quote -v -hk 1F605F33 \
  -bm 21f2d0f812f2492bb376df8fdb62bbeeec86cd81 -pwdk kkk
TPM_Send: GetCapability
TPM_TransmitSocket: To TPM [GetCapability] length=22
[ ... ]
Verification against AIK succeeded
\end{verbatim}

  \subsection{Decryption-based attestation}

    To start, PCR register 12 is extended with the contents of a trusted text
    file.

\begin{verbatim}
cat trusted.txt
THIS IS A TRUSTED FILE!
sha -if trusted.txt -ix 12
SHA1 hash for file 'trusted.txt':
Hash: 8903498ee0749edc141f0655d3b0e3c3239c28b4
New value of PCR: 64995d0f268a5a2788e3567459e205718d057f2d
\end{verbatim}

    Then the new PCR value is used to create and load  a new attestation key.

\begin{verbatim}
createkey -hp 40000000 -kt e -pwdp sss -ok attestation_encryption_key \
  -ix 12 64995d0f268a5a2788e3567459e205718d057f2d
loadkey -hp 40000000 -ik attestation_encryption_key.key -pwdp sss
New Key Handle = C41AD035
\end{verbatim}

    With the new key enabled in the TPM, it's possible to seal the trusted
    file, which binds it to the PCR digest.

\begin{verbatim}
sealfile -hk C41AD035 -if trusted.txt -of trusted.seal
\end{verbatim}

    This yields the encrypted file trusted.seal, which can be unsealed since
    the PCR value in 12 is the same as when it was sealed.

\begin{verbatim}
unsealfile -hk C41AD035 -if trusted.seal -of trusted.unseal
cat trusted.unseal
THIS IS A TRUSTED FILE!
\end{verbatim}

    However, if the PCR value in register 12 does not match, the unsealing of
    the file will fail.

\begin{verbatim}
cat trusted.txt
THIS IS NO LONGER A TRUSTED FILE!
sha -if trusted.txt -ix 12
SHA1 hash for file 'trusted.txt':
Hash: ddf57aca61f940ea08dfd30884ad0fea78e76159
New value of PCR: 49735514f4860c4a330f5869a5e4a3a9d8ed8069
unsealfile -hk C41AD035 -if trusted.seal -of trusted.unseal
Error PCR mismatch from TPM_Unseal
\end{verbatim}

\section{Assignment 9: Your first TPM application}

The sample.c code was modified to produce the final tpm-program which can be
found here: \url{https://pastebin.com/0m2e2GRQ}.

Where the following section is of the most interest:

\begin{lstlisting}
[...]
// Set up parameters for random byte generation.
UINT32 size_random = 32;
BYTE * random_bytes = NULL;

// Generate and print the random bytes.
Tspi_TPM_GetRandom(hTPM, size_random, &random_bytes);
printf("Generating %zu random bytes.\n", size_random);
for (UINT32 i=0; i<size_random; i++) {
    printf("  Random byte #%d: %2x\n", i+1, random_bytes[i]);
}
printf("Done with byte-generation.\n");

// Free random data.
Tspi_Context_FreeMemory(hTPM, random_bytes);
DBG("Free memory allocated to random bytes.\n", result);

// Read values from the first 3 PCR registers.
printf("Printing values of first 3 PCR registers.\n");
BYTE * pcr_data_pointer = NULL;
for (UINT32 i=0; i<3; i++) {
    UINT32 size_pcr_data = 0;
    Tspi_TPM_PcrRead(hTPM, i, &size_pcr_data, &pcr_data_pointer);
    printf("Value in PCR-register#%zu: ", i);
    for (UINT32 j=0; j<size_pcr_data; j++) {
        printf("%2x", pcr_data_pointer[j]);
    }
    printf(".\n");
}
printf("Done printing PCR registers.\n");
[...]
\end{lstlisting}

Where 32 random bytes are generated and printed, and the data from the first 3 PCR-registers
are displayed.

\textbf{To get the code running from newly started TPM-emulator:}

Copy the code from the paste bin into a file called tpmcode.c.

Commands in the connected TSS-command-window.
\begin{itemize}
  \item{forceclear}
  \item{tmp\_setenable -e -f}
  \item{tpm\_takeownership -z -y}
  \item{gcc -o tpmcode tpmcode.c -ltspi -std=c99 -Wall}
  \item{./tpmcode}
\end{itemize}

\end{document}
