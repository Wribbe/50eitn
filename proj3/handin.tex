\documentclass[10pt]{article}
\usepackage{enumitem}
\usepackage{parskip}
\usepackage[T1]{fontenc}
\usepackage[utf8]{inputenc}
%\usepackage{fullpage}

\newcommand{\command}[1]{\texttt{#1}}
\newcommand{\Q}[2]{\textit{\textbf{Q#1}: #2}}
\newcommand{\A}[1]{\textbf{A}: #1}

\begin{document}

  % Title page.
  %------------

  \thispagestyle{empty}
  \vspace*{3cm}
  \begin{center}
    \huge{EITN50 -- Advanced Computer Security} \\
    \vspace{0.3cm}
    \LARGE{TPM - Trusted Platform Module} \\
    \vspace{1cm}
    \large{by: Anonymous} \\
  \end{center}

  % First page
  %-----------

  \newpage

  \section*{Introduction}

    The purpose of the TPM project is to familiarize the students with the
    TPM12 specification through the interaction with a simulated TPM. The more
    specific learning goals are:

    \begin{itemize}
      \item{Introduce TPM usage concept.}
      \item{Understand TPM1.2 key hierarchy.}
      \item{Understand binding, sealing and attestation.}
      \item{Understand key migration.}
      \item{Write a simple TPM application.}
    \end{itemize}

    This report is based on TPM-handout v1.0 (2017-09-17).

  % Set section counter to 2 to match assignment numbers.
  \setcounter{section}{2}

  \section{Assignments}

    \subsection{Setting up the TPM environment}

      Get the virtual machine up and running:

      \begin{enumerate}
        \item{Copy image archives from network drive S:\textbackslash Courses\textbackslash eit\textbackslash EITN50\textbackslash Project - TPM to
          local lab room computer}
        \item{Extract the images using 7-zip or similar.}
        \item{Add the TPM1, TMP2 and TSS images as machines to virtualbox:
          \command{Machine > Add}.}
        \item{Start the TPM1 and TSS virtual machines.}
        \item{Login on TPM1 with user: \emph{pi} and password: \emph{tpm}.}
        \item{Login on TSS with user: \emph{tss} and password: \emph{lab}.}
      \end{enumerate}

      Ensure that the necessary environment variables are set to correct values
      for TPM1:

      \begin{enumerate}
        \item{Open terminal on TPM1.}
        \item{Check IP address with: \command{ifconfig eth0}, should be 10.0.2.14.}
        \item{Check the environment variables of TPM1 with: \command{env | grep
          \textasciicircum TPM}.}
      \end{enumerate}

      The last command should return the following:
      \begin{quote}
        pi@TPM1 ~ \$ env | grep \textasciicircum TPM\\
        TPM\_PORT=6545\\
        TPM\_SERVER\_NAME=localhost\\
        TPM\_PATH=/home/pi/tpm/tpm4720/tpmstate\\
        TPM\_SERVER\_PORT=6545
      \end{quote}

      Ensure that the necessary environment variables are set to correct values
      for TSS:

      \begin{enumerate}
        \item{Open terminal on TSS.}
        \item{Check the environment variables of TSS with: \command{env | grep
          \textasciicircum TCSD}.}
      \end{enumerate}

      Which should return:

      \begin{quote}
        tss@TSS ~/Desktop/share \$ set | grep \textasciicircum TCSD\\
        TCSD\_TCP\_DEVICE\_HOSTNAME=10.0.2.15\\
        TCSD\_TCP\_DEVICE\_PORT=6545\\
        TCSD\_USE\_TCP\_DEVICE=true
      \end{quote}

      The variable TCSD\_TCP\_DEVICE\_HOSTNAME, needs to be changed from *.15 to
      *.14, which can be done by running:

      \command{export TCSD\_TCP\_DEVICE\_HOSTNAME=10.0.2.14}

      And the same goes for TPM\_SERVER\_HOSTNAME:

      \command{export TPM\_SERVER\_HOSTNAME=10.0.2.14}

      Start the emulator on the TPM1 machine:

      \command{tpm\_server}

      and connect to the running instance it on the TSS machine:

      \command{tpmbios} \\
      \command{sudo -E /usr/local/sbin/tcsd -e -f}

      open a second terminal on the TSS machine, this window will be used to
      issue all commands to the remote TPM emulator.

    \subsection{Generating EK, ownership and SRK public key}

      Use the command-window that is open on the TSS instance and run
      \command{createek}. This generates the EK public key and prints
      it to the terminal running the \command{tpm\_server} command
      on the TPM1 computer. \\

      Scrolling through the output yields the following:

      \begin{quote}
        ...\\
        TPM\_RSAGenerateKeyPair: length of n,p,q,d = 256/128/128/256\\
        TPM\_Key\_GenerateRSA: Public key n 9d c7 a2 12\\
        TPM\_Key\_GenerateRSA: Exponent length 3\\
        01 00 01\\
        TPM\_Key\_GenerateRSA: Private prime p cc f1 86 7b\\
        TPM\_Key\_GenerateRSA: Private prime q c5 16 32 ff\\
        TPM\_Key\_GenerateRSA: Private key d 8d 80 b5 01\\
        ...\\
        TPM\_SymmetricKeyData\_SetKeys:\\
        TPM\_SymmetricKeyData\_SetKeys: userKey 48 67 2f 01\\
        ...
      \end{quote}

      which gives:

      \begin{itemize}
        \item{Public key -- 9d c7 a2 12}
        \item{Private key -- 7d 80 b5 01}
      \end{itemize}

      The next step is to take ownership of the simulated TPM. This is done by
      running the following command in the TSS command window:

      \command{takeown -pwdo superhemligt\_o -pwds superhemligt\_s}

      where \textit{superhemligt\_o} is the owner password and
      \textit{superhemligt\_s} is the storage root key password.

      After the ownership is established, the public SRK key can be dumped using
      the following command:

      \command{ownerreadinternalpub -hk 40000000 -of srk.pub -pwdo
      superhemligt\_o}

      Which dumps the public key to the file \textit{srk.pub} and prints:

      \begin{quote}
        \texttt{TPM\_IO\_Write: length 335\\
        \ \ \  BF E7 8E C3 A4 2E 0D D4 00 DE \\
     C7 0F 97 EC D8 C2 55 92 E0 8A 59 1B 92 F8 27 60 \\
     69 58 6A 2A 69 08 67 6A 9D 05 CF 92 70 B7 FF B1 \\ 
     95 47 26 BC 1E D6 86 4F 5A 24 72 CA CE AE 6A 32 \\ 
     F7 11 53 88 25 42 45 CC D0 41 B5 98 B4 F3 67 D4  \\
     01 1B CE B1 3D B6 85 E8 1C 52 E9 71 2A 34 9E 09  \\
     ED D9 75 32 34 E2 00 E6 68 3D 61 7F C8 CA E9 27  \\
     0B 56 04 3B 5A 06 F0 CC 5D EA 07 BC 00 19 D7 A3 \\ 
     00 D8 DF 7F BC D7 59 E1 11 2F C8 53 C2 FD DA 8C \\ 
     66 38 38 11 D4 8E 84 9E 02 65 C0 EA FB EB 39 08  \\
     AC 56 74 FF 3C 16 51 45 C1 49 34 64 6B F8 E1 63  \\
     D0 C0 0D 18 7E 64 E2 E5 18 8D AC 3E 02 89 10 33  \\
     A6 6A 20 9B D6 CB 87 D5 39 DB 53 4C BC F4 B6 C6 \\ 
     8D 6F 57 37 AD E6 B9 4E 40 4A 78 7B C5 7F BE CC \\ 
     BD 30 2E 0E 4E 4F 74 50 34 B9 9F B1 CF 8D B8 E5  \\
     5A E6 4C 47 AB 3C 16 0B AB 55 9B C8 BB 23 1C 5E \\ 
     3E 2C 27 DB 12 B7
       }
      \end{quote}

      to the terminal running the TPM-emulator.
      and adding the \command{-v} flag to the \\
      \command{ownerreadinternalpub} command
      prints the same bytes to the TSS command window.

  \subsection{Key hierarchy}

    \Q{1}{Describe the difference between an identity and signature key.}\\
    \A{
      An identity key is an alias for the Endorsement Key. The alias is used
      to mitigate privacy and security issues. Without the alias is it easy to
      link a single EK to a single user. The alias also avoids reducing the
      entropy of the EK by not using it directly. An AIK (Attestation Identity
      Key) and signing keys are limited in which operations they can be used
      in. It's possible to use the AIK in the \command{TPM\_Quote} and
      \command{TPM\_CertifyKey}, but not in \command{TPM\_Sign}, where a
      signing key should be used. The general difference between identity and
      signature keys is that signing keys are used to sign arbitrary data and
      identity keys are used for remote attestation.
  }

    \Q{2}{Which keys can be used for file encryption?}\\
    \A{Storage keys.}

    \Q{3}{
      There is a key that exists, but using it is not recommended. Which key
      is that, and why does it exist?
    }\\
    \A{
      The EK should not be used directly due to the reasoning in Q1 and
      instead be used to create keys that can't be migrated (and thus can be
      trusted), which are then used directly in place of EK when possible.
    }

    The key hierarchy:
    \begin{quote}
      \texttt{
SRK\ \\
\ \ \ /\ \ \textbackslash \\
H\ \ \ A \\
\ \ \ \ /\ |\ \textbackslash \\
\ \ \ B\ \ F\ \ G \\
\ /\ |\ \textbackslash \\
C\ \ D\ \ E \\
      }
    \end{quote}

\section{Assignment 4: Key Migration}
\subsection{Questions}
\begin{enumerate}
    \item {Is it possible for a migratable key to be the parent of a non-migratable key?} 
	\begin{itemize}
	    \item {Yes, but then the non-migratable key becomes a migratable key.}
	\end{itemize}
 
    \item {Which command is the first to be executed when performing a key migration?} 
	\begin{itemize}
	    \item {TPM\_AuthorizeMigrationKey  $\rightarrow$  TPM\_CreateMigrationBlob  (i. e., authorize migation key cmd)}
	\end{itemize}

    \item {Give a short description of the command TPM\_ConvertMigrationBlob.}
	\begin{itemize}
	    \item {This command enables a migration-blob to be loaded into a TPM as a 'normal', wrapped blob.\\
From TPM part 3 pdf: "Loading one of these wrapped blobs does not require authorization, since correct blobs were created by the TPM under Owner authorization, and unwrapped blobs cannot be used without Owner authorisation." \\
So: unwrapped blobs cannot be used w/o owner auth, wrapped w/o, guess this command makes a blob usable for the "new" TPM, so that it can use the keys in the blob w/o prompting user.}
	\end{itemize}

    \item {Which TPM command loads the migrated keys into the TPM?}
	\begin{itemize}
	    \item {TPM\_LoadKey}
	\end{itemize}

    \item {Is it the TPM or the TSS that handles the transfer of the migration blob? }
	\begin{itemize}
	    \item {The TPM.}
	\end{itemize}
\end{enumerate}

\subsection{Instructions: Key migration in the TPM emulator}
\subsection{Questions}

\section{Assignment 5: Extending values to PCRs}
\subsection{Questions}
\subsection{SHA-1 calculation and PCR extending using the TPM emulator}
\section{Assignment 6: File encryption}
\subsection{Questions}
\subsection{Data binding using the TPM emulator}
\subsection{Data sealing using the TPM emulator}
\section{Assignment 7: TPM Authentication}
\subsection{Questions}
\subsection{Authentication exercises}
\section{Assignment 8: Attestation}
\subsection{Signature based attestation}
\subsection{Decryption-based attestation}
\section{Assignment 9: Your first TPM application}

\end{document}
