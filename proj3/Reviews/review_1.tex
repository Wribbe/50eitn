\documentclass{article}

\usepackage[T1]{fontenc}
\usepackage[utf8]{inputenc}
\usepackage{parskip}
\usepackage{varwidth}
\usepackage{ifthen}
\usepackage{amssymb}
\usepackage{color}
\usepackage[Q=yes]{examplep}
%\usepackage{fullpage}


% NOTE:
% -----
%
%     Only project reports of TPM and Camera project are submitted through this page.
%
%     All submissions must be anonymous, both your solutions and your gradings
%     of other solutions.
%
%     Your answer sheet must also include the question. Copy and paste the
%     questions from your personal problems sheet.
%
%     You may replace your uploaded file as many times as you want before the
%     deadline.
%
%     It is not possible to upload documents after the deadline.
%
%     You will be able to see the projects to grade after the solutions deadline.
%
%     You will be able to see your grade result after the result deadline, but
%     you can always see your uploaded gradings.
%
%     Uploaded files must be PDF.
%
% Think about the following when you do the grading:
% --------------------------------------------------
%
%     If you deduct points, motivate this such that the person you grade
%     understands why points are deducted.
%
%     If you give full points, just say e.g., "Yes, this agrees with the
%     information on page XX or slide XX or http://...". Be specific about the
%     source which you have used to verify the answer.
%
%     Some answers can not be immediately found any material but are instead
%     based on understanding a certain concept. In this case, make an effort to
%     verify the correctness by understanding the concept yourself.
%
%     If you are unsure about the grading, say why but be generous when giving
%     points.
%
% Post-grading procedures:
% ------------------------
%
%     You are welcome to discuss the grading of your project with Ben or Jonathan
%     and we are willing to modify your grade if you motivate why (and the
%     motivation is reasonable). In order to facilitate a fruitful discussion
%     this will be done during the office hours written on the webpage.
%
%     Discussing your result with Ben does not have a deadline and can be done at
%     any time during the course (but please respect the office hours).  If your
%     score is modified it will not be seen on this page, but can be seen on the
%     result summary on the course webpage.

\newcommand{\grade}[5]{
  % Heading with identifier of grading criterion and max points.
  \textbf{Grading criterion; #1 (#3p):}\textit{#4}

  % Print reason behind grade.
  #5
  % Print awarded score.
  \hfill
  \begin{varwidth}{\linewidth}
    \textbf{Points: #2/#3}
  \end{varwidth}
}

\newcommand{\QAN}[5]{
  % Question par.
  \textit{Q#1}:#3  \\
  % Answer part.
  \ifthenelse{\equal{#2}{C}}{
    \def \grading {\Large\textcolor{green}{\checkmark}\normalsize}
  }{
    \def \grading {\Large\textcolor{red}{\textbf{x}\hspace{0.142cm}}\normalsize}
  }
  \textit{A:}#4\hfill\grading \\
  % Note part.
  \textit{Note:}#5
}

\newcommand{\escape}[1]{\PVerb{#1}}

\begin{document}
  % Title.
  \begin{center}
    \huge{Review -- Project: TPM} \\
    \vspace{0.2cm}
    \Large{\today}
    \vspace{0.4cm}
    \hrule
  \end{center}
  % End title.

  \grade{assignment 1}{5}{5}
  {
    Report describes the necessary steps to repeat the group's
    experimental setup.
  }
  {
    The reproducibility is high, most steps are explained in detail.
  }

  \grade{generating EK}{1}{2}
  {
    Report should contain a dump EK and description how it was obtained.
  }
  {
    Contains both. However, there is a byte sequence in the beginning of the dump that does not change between different instances of TPMs. This leads one to belive that this sequence is not part of the EK, but some sort of preamble or overhead. 
The byte sequence follows here: 
\begin{quote}
00 C4 00 00 01 3A 00 00 00 00 00 00 00 01 00 03 \\
00 01 00 00 00 0C 00 00 08 00 00 00 00 02 00 00 \\
00 00 00 00 01 00
\end{quote}

It would have been more correct if the actual key part would have been marked in the dump.
  }

  \grade{dump SRK public key}{2}{2}
  {
    Report should contain a dump SRK pubkey and description how it was
    obtained.
  }
  {
    Contains both.
  }

  \grade{Key hierarchy questions}{4}{6}
  {
    Each correct answer to the questions equals 2 points.
  }
  {
    \QAN{1}{C}{
      Difference between identity and signature keys?
    }
    {
      An identity key must meet minimum security requirements and needs the SRK as its
parent while the signature key does not need any specific security and can be created
further down the key hierarchy.
    }
    {
      Correct. \escape{http://opensecuritytraining.info/IntroToTrustedComputing_files/Day1-7-tpm-keys.pdf} , slide 8.
    }

    \QAN{2}{F}{
      Which keys can be used for file encryption?
    }
    {
      All keys can theoretically be used for file encryption but, since we do not want to
compromise our key hierarchy and provide adequate security, the best key to use is a
storage key whose main purpose is to encrypt the file contents.
    }
    {
      It is correct that all of the key types could theoretically in isolation be used for encryption, but in a TPM system that is not allowed; the documentation on the TPM places restrictions on using certain key types for certain operations, as well as almost full restriction on EK. TPM key types (L4 -- 59), and:

\begin{quote}
"The EK is only available for two operations:
establishing the TPM Owner and establishing Attestation Identity Key (AIK) values and
credentials. There is a prohibition on the use of the EK for any other operation."
	\hfill
	-- (TPM Part1, page 58)
\end{quote}

Correct that a storage key is best, but motivation is not complete or correct.
    }

    \QAN{3}{C}{
      There is one type of key that exists, but is not recommended. Which key
      is that, and why does it exist?
    }
    {
     Legacy key, which can be used both for signing and encryption although not recommended.
They have lower security in order to provide backwards compatibility with
older standards.
    }
    {
      Combination of ''There was a TPM version 1.0 but we can today forget
      about that version'' (L4 -- 53) and ''Legacy: [...] (compatible with TPM
      v1)'' (L4 -- 59).
    }
  }

  \grade{Key hierarchy -- Possible combinations}{2}{2}
  {}
  {
    \QAN{}{C}{
      Are all combinations possible? if not, why?
    }
    {
      Note that the creation
      of key C failed since all children of a migratable key also needs to be migratable.
     % The key named C could not be created. The reason for this is that a non
      %migratable key can not be a child of a migratable key.
    }
    {
      Creating non-migratable keys with migratable parent as base should return
      \texttt{TPM\_INVALID\_KEYUSAGE} (TPM-part3 -- p.74).
    }
  }

  \newpage

  \grade{Key hierarchy -- Correct drawing}{1}{2}
  {
    Drawing / representation of correct hierarchy.
  }
  {
    Representation is almost correct, but includes the C - key.
  }

  \grade{Key migration - questions}{7}{10}
  {
    Each correct answer gives 2 points.
  }
  {
    \QAN{1}{C}{
      Is it possible for a migratable key to be the parent of a non-migratable
      key?
    }
    {
     No, since when a migratable parent key is exported, all its children needs to be able
to tag along. Having the child non-migratable would then cause a contradiction.
    %  No, it is not possible for a migratable key to be the parent of
      %non-migratable keys. When a key is migrated, all of its children are
      %migrated as well. Thus, all children of a migratable key have to be
      %migratable
    }
    {
      Creating non-migratable keys with migratable parent as base should return
      \escape{TPM_INVALID_KEYUSAGE} (TPM-part3 -- p.74).
    }

    \QAN{2}{F}{
      Which command is the first to be executed when performing a key
      migration?
    }
    {
On the TPM you are about to import into [..] you should create a
new migrateable storage key that can then be used with TPM1 to encrypt keys you
want to export with the command \escape{TPM_CreateMigrationBlob}.      
%\escape{TPM_CreateMigrationBlob} is the command that implements the first
      %step in the process of moving a key to a new parent or platform.
    }
    {
     % Correct according to description (TPM-part2 -- p.85).
 Not correct according to description (TPM-part3 -- p.85):

\begin{quote}
The TPM Owner does the selection and authorization of migration public keys at any time
prior to the execution of \escape{TPM_CreateMigrationBlob} by performing the
\escape{TPM_AuthorizeMigrationKey} command.
\end{quote}

According to this \escape{TPM_CreateMigrationBlob} is the second command to be executed, after \escape{TPM_AuthorizeMigrationKey}. This is also supported by \escape{https://shazkhan.files.wordpress.com/2010/10/http__www-trust-rub-de_media_ei_lehrmaterialien_trusted-computing_keyreplication_.pdf}, page 4.
    }

    \QAN{3}{F}{
      Give a short description of the command \escape{TPM_ConvertMigrationBlob}
    }
    {
This command takes a migration blob and decrypts it to a normal wrapped blob
which is then possible load into the TPM using the \escape{TPM_LoadKey} function. Note that
the command migrates private keys only. The migration of the associated public keys
is not specified by TPM because they are not security sensitive.
      %This command takes migration blob and creates a normal wrapped blob. The
      %migrate blob must be loaded into the TPM using the normal
      %\escape{TPM_loadKey} function.
    }
    {
      One point deduction due to \escape{TPM_loadKey} being deprecated and \\
      \escape{TPM_loadKey2} should be used instead (TPM-part3 -- p.318).
    }

    \QAN{4}{F}{
      Which TPM command load the migrated keys into the TPM?
    }
    {
After the migration blob has been converted, the command to load the wrapped key
into the TPM is \escape{TPM_LoadKey}.
      %\escape{TPM_loadKey}
    }
    {
      Same as above, \escape{TPM_loadKey} is deprecated, should use
      \escape{TPM_loadKey2} instead.
    }

    \QAN{5}{C}{
      Is it the TPM or the TSS that handles the transfer of the migration blob?
    }
    {
It is the TSS, since TPM:s have no ability to communicate directly with another
TPM.
      %The TSS handles the transfer of the migration blob.
    }
    {
      Correct. Since the TPM only handles conversion of input and output data, the
      transfer of the resulting/necessary data is handled by the software stack
      (TSS).
    }
  }

  \grade{Key migration -- migration \& documentation}{2}{2}
  {
    Do the key migration specified in the project instructions and document it
    (Q1).
  }
  {
    Good.
  }

  \grade{Key migration -- remaining questions}{1}{4}
  {}
  {
    \QAN{2}{F}{
      When do you use a key of type \escape{TPM_KEY_USAGE = TPM_Migrate}?
    }
    {
\escape{TPM_KEY_USAGE} can have the value \escape{TPM_KEY_MIGRATE}, which is used when there is a
need for a migration authority (we did not find any \escape{TPM_migrate} command).
      %When using migration authority.
    }
    {
      The \escape{TPM_KEY_USAGE = TPM_Migrate} is used to restrict a specific
      key in such a way that it can only be used in the \escape{TPM_MigrateKey}
      function. Since this function performs the function of a migration
      authority with limited knowledge about the key, the physical security of
      the executing system is assumed to be high (TPM-part3 p.93).
    }

    \QAN{3}{F}{
      What is the rewrap option of the \escape{migrate} command used
      for?
    }
    {
This is used when migration authority is needed.
      %The rewrap option is use to directly move a key to a new parent.
    }
    {
Not correct. It allows a key to be directly moved to another parent, either in the same or another TPM. The re-wrap flag tells the TPM to re-wrap
      (decrypt$\rightarrow$encrypt) the key with a new parent, which enables
      that parent to load the key as a normal encrypted element (TPM-part3
      p.85).
    }
  }

  \grade{Extending values to the PCRs -- Questions}{4}{4}
  {}
  {
    \QAN{1}{C}{ % Question
      Describe one TPM command that can be used to extend the SHA-1 digest to a
      PCR.
    }{ % Answer
      \escape{TPM_SHA1CompleteExtend}
“This capability terminates a pending SHA-1 calculation and EXTENDS the result
into a Platform Configuration Register using a SHA-1 hash process.”
%The \escape{TPM_Extend} command can be used to extend the SHA-1 digest to
     % a PCR by adding a new measurement to a PCR.
    }{ % Note
      Correct (TPM-part3 p.160).
    }

    \QAN{2}{C}{ % Question
      Describe one TPM command that can be used to read a PCR value.
    }{ % Answer
\escape{TPM_PCRRead}
“The \escape{TPM_PCRRead} operation provides non-cryptographic reporting of the contents of
a named PCR.”      
%The \escape{TPM_PCRread} command can be used to read a PCR value.
    }{ % Note
      Correct (TPM-part3 p.162)
    }
  }

  \grade{Extending the PCRs -- PCR 'overflow'}{2}{2}
  {
    Run \escape{sha -if <filename> -ix <PCR index>} on a large file and show
    the result.
  }{
The command was run on the \escape{tpmbios} file and adequate prints have been included in the report.
%    The figure shows that the command has been run, but since it is run on \\
    %\escape{text.txt} and there is no indication of then file size of the file
    %it's hard to figure out if the run illustrated the functionality or not.
  }

  \grade{File encryption -- Questions}{6}{6}
  {}
  {
    \QAN{1}{C}{ % Question
      Why is \escape{TSS_Bind} a TSS command, and not a TPM command?
    }{ % Answer
The public part of the binding key pair is used for encrypting data. This should be
possible to do from anywhere, so the need to load the private key is unnecessary.
Once the data is bound, only the one with control over the private part should be
able to decrypt it, which is why the private key needs to be loaded for decryption to
be possible. This only requires the unbind operation to be a TPM command.
%      The encryption is done outside of the TPM and is therefor not a TPM
    %  command.
    }{ % Note
      Correct (L5 - p.11).
    }

    \QAN{2}{C}{ % Question
      Give some difference between data binding and data sealing.
    }{ % Answer
\begin{itemize}
	\item “Binding”: we can encrypt data on another computer, and decryption can only be
done on the computer which has the TPM with the private key. 
	\item “Sealing”: binds data to a certain value of the PCR and a key that is not migratable.
Then only this specific TPM can decrypt (unseal) the data, and even then only if the
PCR value is the same as when encryption happened (sealing). 
\end{itemize}
%      Data binding uses a symmetric key for encryption and data sealing uses
    %  asymmetric encryption.
    }{ % Note
      Correct.
    }

    \QAN{3}{C}{ % Question
      Can a key used for data sealing be migrated to another TPM?
    }{ % Answer
No since, the key used for the sealing is required to be non-migratable.      
%No, because sealing data to one TPM platform makes it illegal to migrate
      %a key to another TPM.
    }{ % Note
      Correct.
    }
  }

  \grade{TPM -- Data binding}{4}{4}
  {}
  {
    \QAN{1}{C}{ % Question
      Why does the key have to be loaded inside the TPM when decrypting, but
      not when encrypting?
    }{ % Answer
Since we not only need secure keys but also a secure platform in order to correctly
decrypt the file, i.e. not possible without a authorized TPM. This is implemented
in parts by requiring the private key to be descended from SRK which in turn is
related to the unique secret of the TPM chip, the decryption needs parts of the secret
only known to the chip manufacturer. [..] The key is also encrypted via the need for
password previously set by the TPM.
%      When encrypting, only the public key is needed, while decryption uses the
    %  private key, which is not accessible outside of the TPM.
    }{ % Note
	Not a very clear description, but correct. 
    }

    \QAN{2}{C}{ % Question
      Migrate the binding key to TPM2 and see if the file can be decrypted
      there.
    }{ % Answer
      %To be able to decrypt the binding key is needed. Since the binding key
      %was migrated to TPM2, the file was able to be decrypted there too.
    }{ % Note
      Should be possible, and it worked.
    }
  }

  \grade{TPM -- Data sealing}{5}{5}
  {}
  {
    \QAN{1}{C}{ % Question
      Test if you can do a sealing with a legacy key, a binding key or a
      signing key. If not, why?
    }{ % Answer
As can be seen, it is only the non-migratable storage key that can be used for
sealing. [..] The key needs to be non-migratable in order to be used by the sealfile command.
It was however possible to use migratable keys when using bindfile command,
since with bind we only encrypt with the key itself. Contrast this with sealing
were we also bind the encryption with the PCR values of the TPM. Therefore it is
impossible to seal a file with migratable keys as they would be useless for decryption
since some PCR values can’t be extracted out of the TPM.
      %It was not possible to do a sealing with a legacy key, binding key or
      %signing-key. The reason for this is that sealing only can be done with
      %storage keys.
    }{ % Note
      \begin{quote}
        If the keyUsage field of the key indicated by the keyHandle does not
        have the value \escape{TPM_KEY_STORAGE} the TPM must return the error
        code \escape{TPM_INVALID_KEYUSAGE}.

        \hfill
        -- Documentation of \escape{TPM_Seal} (TPM-part3 p.63)
      \end{quote}
    }

    \QAN{2}{C}{ % Question
      Now migrate the storage key to TPM2 and see if you can unseal the file
      there too. Explain what you observe.
    }{ % Answer
	
 %     The storage key is not migratable and could therefor not be migrated to
     % TPM2. Since TPM2 did not have access to the storage key, unsealing was
     % not achieved.
    }{ % Note
	An attempt was made, but did not work, it is described why ("As can be seen, it is only the non-migratable storage key that can be used for
sealing.")

      \begin{quote}
        If the keyHandle points to a migratable key then the TPM MUST return
        the error code \escape{TPM_INVALID_KEYUSAGE}.

        \hfill
        -- Documentation of \escape{TPM_Seal} (TPM-part3 p.63)
      \end{quote}
    }
  }

  \grade{TPM -- Authentication}{6}{6}
  {}
  {
    \QAN{1}{C}{ % Question
      In the above, could the \escape{verifyfile} command been done by another
      TPM?
    }{ % Answer
Yes, since the signing is done by the private key, the public part can be used for
verifying the private signing. And since that part is public, any TPM can use it.
%      Yes it is possible because only the public key is used.
    }{ % Note
      The usage string for verifyfile agrees with only using public part:
      \escape{verifyfile [-ss info|der] -is <sig file> -if <data file> -ik
      <pubkey file (.pem)>}.
    }

    \newpage

    \QAN{2}{C}{ % Question
      Which TPM command is used to decrypt the file?
    }{ % Answer
“\escape{TPM_UnBind} takes the data blob that is the result of a \escape{Tspi_Data_Bind} command
and decrypts it for export to the User." [..]
The binding operation is undone by unbind. This requires the private part of the
keypair and is done inside the TPM.
%      The command that is used to decrypt files is \escape{TPM_UnBind}.
    }{ % Note
      Same decryption command as earlier.
    }

    \QAN{3}{C}{ % Question
      Can the decryption based authentication be done by using data sealing
      instead of binding?
    }{ % Answer
Not really. Sealing limits the encryption and decryption to only one specific TPM.
No other TPM should be able to encrypt or decrypt it which means the only external
party that the TPM can authenticate towards is itself, which is kind of moot.
%      No it can not because the TPMs have different PCR values.
    }{ % Note
      %Sealing can work to identify a TPM. If a user encrypts a piece of data
      %with the public part of the key (and PCR values for that TPM is included), only the TPM that has the correct
      %private part and is in the correct measured configuration state can
      %decrypt it and return the data. Which in turn proves that it's the same
      %TPM, with the same configuration. 

From \escape{http://opensecuritytraining.info/IntroToTrustedComputing_files/Day2-1-auth-and-att.pdf} one can see that
"If you want.. \escape{Decryption-based Authentication}, Use key type.. \escape{Binding} with..  \escape{Bind}" (paraphrase of table on Summary page).
    }
  }

  \grade{Signing}{2}{2}
  {
    Use \escape{signfile} and \escape{verifyfile} to sign and verify a text
    file.
  }
  {
    Presented commands are correct.
  }

  \grade{Encryption}{2}{2}
  {
    Use \escape{bindfile} and \escape{unbindfile} to encrypt and decrypt a
    text file.
  }
  {
    Presented commands are correct.
  }

  \grade{Attestation -- signature}{2}{2}
  {
    Use the \escape{identity} and \escape{quote} commands to create an AIK and
    use that to quote a PCR value.
  }
  {
    Provided commands are correct.
  }

  \grade{Attestation -- decryption}{3}{3}
  {
    Use the \escape{createkey -ix}, \escape{sealfile} and \escape{unsealfile}
    commands to bind a hash-digest for a file to a storage key. Unseal the
    original file, which should work. Modify the file and extend the PCR with
    the new file, the unsealing should fail.
  }
  {
    The correct commands are provided.
  }

  \grade{Creating a TPM program}{4}{4}
  {
    Provide correctly working program with source code and documentation to
    repeat the work.
  }
  {
    Program is included, should work.
  }

  \textbf{Total: 65/75} %note: ben says 74 points, but after counting ~5 times I still get 75 /raz

One could see that several parts of the report were rewritten parts from the project description. On the other hand, this together with some augmenting explanations, clear commands and printouts made for a very readable report that left few questions as to reproducibility. Good job.

\end{document}
