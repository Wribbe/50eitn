\documentclass[10pt]{article}
\usepackage{enumitem}
\usepackage{parskip}
\usepackage[T1]{fontenc}
\usepackage[utf8]{inputenc}
\usepackage{listings}
\usepackage{tikz}
\usepackage{enumitem}
\usepackage[Q=yes]{examplep}
\usepackage{graphicx}
\usepackage{multicol}
\usepackage[hidelinks]{hyperref}
\usepackage{color}
\usepackage{wasysym}

\newcommand{\command}[1]{\texttt{#1}}
\newcommand{\e}[1]{\PVerb{#1}}
\newcommand{\comm}[1]{{\leavevmode\color{gray}#1}}
\newcommand{\todo}[1]{
  \begin{center}
    [\textcolor{red}{\textbf{\textit{#1}}}]
  \end{center}
}

% Check list evnironment.
\newenvironment{checklist}{%
  \begin{list}{}{}%
  \let\olditem\item
  \renewcommand\item{\olditem -- \marginpar{$\Box$} }
  \newcommand\checkeditem{\olditem -- \marginpar{$\CheckedBox$} }
}{%
  \end{list}
}


\begin{document}

  % Title page.
  %------------

  \thispagestyle{empty}
  \vspace*{3cm}
  \begin{center}
    \huge{EITN50 -- Advanced Computer Security} \\
    \vspace{0.3cm}
    \LARGE{Trusted Camera} \\
    \vspace{1cm}
    \large{by: \\ \vspace{0.2cm}
	\textit{adsec03} \\
        Stefan Eng \texttt{<atn08sen@student.lu.se>} \\
        Rasmus Olofzon \texttt{<muh11rol@student.lu.se>}
        } \\
  \end{center}

  % First page
  %-----------

  \newpage

  \textbf{Parts of camera:}
  \begin{itemize}
    \item{PCB board.}
    \item{Flash memory.}
    \item{RAM.}
    \item{ARM TrustZone CPU}
    \item{TPM}
    \item{JTAG debug interface.}
    \item{LTE subsystem:}
    \begin{itemize}
      \item{USIM card reader.}
      \item{Specific flash memory to hold ID.}
    \end{itemize}
    \item{Battery holding 3 days operational reserve power.}
  \end{itemize}

  \textbf{Report objects:}
  Come up with a architecture/design that meets following requirements:
  \begin{itemize}
    \item{Security.}
    \item{Production.}
    \item{Maintenance.}
  \end{itemize}
  With motivations!

  \section{General notes}

    \subsection{Acronyms}

    \begin{description}[style=multiline,leftmargin=1.7cm]
      \item[CC]{
          (Common Criteria) Secure systems evaluation criteria adopted by
          $\sim$25 countries.
      }
      \item[TOE]{
          (Target Of Evaluation) The system submitted for evaluation.
      }
      \item[ST]{
          (Security Target) The set of security requirements used as the basis
          for a security evaluations.
      }
      \item[EAL]{
          (Evaluation Assurance Level) The evaluation level that is being
          targeted (there are different levels that can be reached).
      }
      \item[TSF]{
          (TOE Security Functions) The combination of software and hardware
          that is necessary to enforce the chosen policy.
      }
      \item[PP]{
          (Protection Profiles) Type of Common Criteria evaluation. Consists of
          implementation-independent security requirements.
      }
      \item[SFR]{
          (Security Functional Requirements) Describes a functional requirement
          (input $\rightarrow$ behaviour $\rightarrow$ output) related to
          security.
      }
    \end{description}

    \subsection{CC evaluation types}

      There are two types of Common Criteria evaluations.

      \subsubsection{Protection profile evaluation}

        The evaluation is concerned with a set of implementation-independent
        requirements that are established for a category of similar products
        and systems. A typical protection profile includes:
        \begin{itemize}
          \item{Introduction containing a description and overview of the
            targeted system.}
          \item{Description of the products or systems in question.}
          \item{Description of the security environment in relation to the
            products or systems.}
          \item{Relevant security objectives.}
          \item{Requirements on IT security.}
          \item{Rationale behind the grouping.}
        \end{itemize}
        Example of protection profile groupings are; antivirus on workstations,
        biometrics, firewalls and intrusion detection systems.

      \subsubsection{Evaluation against a security target}
      Sources:
      \begin{itemize}
        \item{\url{https://en.wikipedia.org/wiki/Security_Target}}
        \item{\e{slides_on_cc.pdf}}
      \end{itemize}

      \comm{
        \textbf{EAL1: Functionally Tested}

        EAL1 is applicable where some confidence in correct operation is
        required, but the threats to security are not viewed as serious. It
        will be of value where independent assurance is required to support the
        contention that due care has been exercised with respect to the
        protection of personal or similar information. EAL1 provides an
        evaluation of the TOE (Target of Evaluation) as made available to the
        customer, including independent testing against a specification, and an
        examination of the guidance documentation provided. It is intended that
        an EAL1 evaluation could be successfully conducted without assistance
        from the developer of the TOE, and for minimal cost. An evaluation at
        this level should provide evidence that the TOE functions in a manner
        consistent with its documentation, and that it provides useful
        protection against identified threats.

        \textbf{EAL2: Structurally Tested}

        EAL2 requires the cooperation of the developer in terms of the delivery
        of design information and test results, but should not demand more
        effort on the part of the developer than is consistent with good
        commercial practice. As such it should not require a substantially
        increased investment of cost or time.  EAL2 is therefore applicable in
        those circumstances where developers or users require a low to moderate
        level of independently assured security in the absence of ready
        availability of the complete development record. Such a situation may
        arise when securing legacy systems.
      }

        The security target contains a list of implementation-specific security
        requirements that should be met by the system or product in question
        together with the steps taken to meet those requirements. The document
        can also reference relevant protection profiles that are fulfilled.
         A typical security target includes:

        \begin{itemize}
          \item{Introduction of what the target does, key features and purpose.}
          \item{Description of threats and assumptions present in the
            operational environment.}
          \item{Security objectives specifying solutions to the perceived
            threats in both the target and its environment together with the
            rationale these were chosen.}
          \item{Definition of functional and assurance security requirements
            together with the reason these are sufficient and necessary.}
          \item{Target summary specification that enables readers to get a
            general understanding of how the target is structured and
            implemented. This section also provides a high-level view of how
            each security functional requirement is solved through developed
            software.}
        \end{itemize}

  \section{Product description}

    \comm{Describe the product and point out which deign requirements that have
    been considered. (Which requirements? Software, hardware, both?)}

    \textbf{Functional requirements:}
    \begin{checklist}
      \item{SRTP stream for data.}
      \item{Possible to update firmware of camera securely.}
      \item{Embed cryptographically strong identifier during production.}
      \item{Flash storage for configurations and recording of 60 min + loop of 60 minutes.}
      \item{Repair/Maintenance personell should not have access to stored data.}
      \item{Possible for user to wipe data on decommission.}
      \item{GUI with information about attestation, firmware, and hash of key
        that protects RTSP stream and stored video.}
      \item{Tamper sensor in camera housing.}
      \item{Possible to patch hardware errors with software and ROM path
        functionality.}
    \end{checklist}

    \todo{0.5 page}

    \subsection{Security assumptions}

      \comm{
        List all the assumptions that affect any of the following:
        \begin{itemize}
          \item{Security.}
          \item{Maintenance.}
          \item{Production costs.}
        \end{itemize}
      }

      \todo{0.5 page}

  \section{Architectural overview}

    \comm{%
      Short high-level architectural overview (guessing hardware?) that
      describes how the design is structured. Motivate all design-choices,
      don't go into to much detail, the level of the description is the
      important part. The design should be presented in such a way that it is
      easy to understand. This section also includes a half page illustration
      of the structure together with the Target Of Evaluation (TOE) and
      Security Target (ST).

      The main point is to exercise technical writing ability. Said description
      should only include details that are important from a security
      perspective. The intended audience is a student that has completed the
      basic computer security course.
    }

    \todo{2-3 pages}

  \section{Security evaluation}

  \textbf{Threats to consider:}
  \begin{checklist}
    \item{Avoid bugs in the (remote) management that would allow someone to
      insert executable foreign code.}
    \item{Should not be possible to load unauthorized firmware.}
    \item{Don't allow loading of SRTP keys for a incorrect receiver.}
    \item{Include safeguards concerning removing flash memory and loading it
      with altered configurations or code.}
    \item{Deploy routines that circumvents dishonest repair personnel.}
    \item{Secure JTAG debug interface against misuse.}
    \item{Mitigate damage due to lost key or camera.}
  \end{checklist}

  \comm{%
    The second part is a short Security evaluation of your design that explains
    what types of protection you use and why. The list of security
    consideration should list all attack types and major security issues you
    can think of and explain if, why and/or how it is or is not applicable to
    your system.  For each attack/issue, explain clearly and concisely what you
    have done to protect your solution.  Or briefly explain why no protection
    is needed. Your report should clearly contain a description that gives the
    motives of your security related design choices.

    Do not forget the consequences for design choices for the manufacturing and
    repair costs.  Your evaluation should contain a table which row wise lists
    the threats that you considered and
  }

    \todo{2-3 pages + summery}

  \section{Peer reviews}
    \todo{2 pages}

  \section{Improvement sheet}
    \todo{1-2 pages}

\end{document}
