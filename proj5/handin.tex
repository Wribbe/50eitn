\documentclass[10pt]{article}
\usepackage{enumitem}
\usepackage{parskip}
\usepackage[T1]{fontenc}
\usepackage[utf8]{inputenc}
\usepackage{listings}
\usepackage{tikz}
\usepackage{enumitem}
\usepackage[Q=yes]{examplep}
\usepackage{graphicx}
\usepackage{multicol}
\usepackage[hidelinks]{hyperref}
\usepackage{color}
\usepackage{wasysym}
\usepackage{listings}

\newcommand{\command}[1]{\texttt{#1}}
\newcommand{\e}[1]{\PVerb{#1}}
\newcommand{\comm}[1]{{\leavevmode\color{gray}#1}}
\newcommand{\todo}[1]{
  \begin{center}
    [\textcolor{red}{\textbf{\textit{#1}}}]
  \end{center}
}
\newcommand{\threat}[3]{\item{\textbf{T.#1} \hfill \textbf{#2} \\ #3}}
\newcommand{\objective}[3]{\item{\textbf{O.#1} \hfill \textbf{#2} \\ #3}}
\newcommand{\assumption}[3]{\item{\textbf{A.#1} \hfill \textbf{#2} \\ #3}} %These could absolutely be done better, but these will do til we've decided on a format
\newcommand{\escape}[1]{\PVerb{#1}}

% Check list evnironment.
\newenvironment{checklist}{%
  \begin{list}{}{}%
  \let\olditem\item
  \renewcommand\item{\olditem -- \marginpar{$\Box$} }
  \newcommand\checkeditem{\olditem -- \marginpar{$\CheckedBox$} }
}{%
  \end{list}
}

\lstdefinestyle{customc}{
  belowcaptionskip=1\baselineskip,
  breaklines=true,
  frame=L,
  xleftmargin=\parindent,
  language=C,
  showstringspaces=false,
  basicstyle=\footnotesize\ttfamily,
  keywordstyle=\bfseries\color{green!40!black},
  commentstyle=\itshape\color{purple!40!black},
  identifierstyle=\color{blue},
  stringstyle=\color{orange},
}

\begin{document}

  % Title page.
  %------------

  \thispagestyle{empty}
  \vspace*{3cm}
  \begin{center}
    \huge{EITN50 -- Advanced Computer Security} \\
    \vspace{0.3cm}
    \LARGE{Trusted Camera} \\
    \vspace{1cm}
    \large{by: \\ \vspace{0.2cm}
	\textit{adsec03} \\
        Stefan Eng \texttt{<atn08sen@student.lu.se>} \\
        Rasmus Olofzon \texttt{<muh11rol@student.lu.se>}
        } \\
  \end{center}

  % First page
  %-----------

  \newpage

  \section{Product description}

    \comm{Describe the product and point out which design requirements that have
    been considered. (Which requirements? Software, hardware, both?)}

    \textbf{Design requirements:}
    \begin{checklist}
      \item{
        SRTP stream for data.

        This should be quite simple? ''Just'' use a library and create a
        stream? Link to open cisco implementation:
        \url{https://github.com/cisco/libsrtp}
      }
      \item{
        Possible to update firmware of camera securely.

        Assuming following types of firmware updates; local, remote pull or
        remote push. Ensure that the software is legit through signing.
        Integrity protection through MAC. New firmware installation should
        include some kind of TPM-involved sealing to ensure that the configuration
        has not been tampered with. Attestation.

        Questions: \\
        -- Which key to sign with? \\
        A: \\
	Note: The hard-to-get-rid-of problem of always needing to store a key somewhere, somehow. Lots of solutions end up 'just' adding complexity 
	and still not solve the actual problem. \\
        -- Which key to create MAC with? \\
        A: \\
        -- How to involve TPM-sealing on upgrade? Need to seal information that
        is necessary for the fw-upgrade.\\
        A: \\

      }
      \item{Embed cryptographically strong identifier during production. \\
	TPM EK?}
      \item{Flash storage for configurations and recording of 60 min + loop of 60 minutes.}
      \item{Repair/Maintenance personel should not have access to stored data.}
      \item{Possible for user to wipe data on decommission.}
      \item{GUI (doen't need to be GUI?) with information about attestation, firmware, and hash of key
        that protects SRTP stream and stored video.}
      \item{Tamper sensor in camera housing. \\
	Tamper resistance can be realised with such an 'easy' method as adding a pushbutton on the PCB and connecting this to a plastic rod in the housing of the camera,
	which connects on one end to the PB and on the other end to the battery hatch (which means PB is a normally-closed circuit). This has the functionality that if 
	the battery hatch is opened (which needs to be done in order to unscrew screws needed to access PCB and other internal components) 
	the PB is no longer pressed down, and this sends an interrupt with a callback that signals 'tampering is present'. \\
	Another part of tamper-sensor should be that if connection to power source is lost, this should notify (? service personnel?). 
	Could also mount camera on wall and make cables go straight in the wall, and a tamper switch to notice if camera has been pulled off wall.\\
	Note: Discussion here probably belongs more in Architectural Overview part of report.}
      \item{Possible to patch hardware errors with software and ROM path
        functionality.}
    \end{checklist}

    \todo{0.5 page}

    \subsection{Security assumptions}

      \comm{
        List all the assumptions that affect any of the following:
        \begin{itemize}
          \item{Security.}
          \item{Maintenance.}
          \item{Production costs.}
        \end{itemize}
      }
	\textbf{Assumptions:}
     \begin{itemize}
	\assumption{LOCATION}{Mounting place}{Mounted on outside of building of a tech company.}
	\assumption{CORRECT\_CONFIG}{Correct configuration}{It is assumed that SRTP is used and that authentication and encryption are enabled.}
	\assumption{TIMELY\_MAINT}{Maintenance schedule followed}{It is assumed that maintenance personnel follows the maintenance schedule
	and that no deterioration of product due to neglected maintenance occurs.}
     \end{itemize}

      \todo{0.5 page}

  \section{Architectural overview}

    \comm{
      Short high-level architectural overview (guessing hardware?) that
      describes how the design is structured. Motivate all design-choices,
      don't go into to much detail, the level of the description is the
      important part. The design should be presented in such a way that it is
      easy to understand. This section also includes a half page illustration
      of the structure together with the Target Of Evaluation (TOE) and
      Security Target (ST).

      The main point is to exercise technical writing ability. Said description
      should only include details that are important from a security
      perspective. The intended audience is a student that has completed the
      basic computer security course.
    }

	For the illustration, probs smth like page 15 of \escape{https://www.commoncriteriaportal.org/files/ppfiles/pp0089b_pdf.pdf}
  \subsection{TOE}
    \begin{checklist}
	\objective{SECURECOMMS} {Protected Communications} {
	Add stuff here please. Mention SRTP.}
    \end{checklist}

    \todo{2-3 pages}

  \section{Security evaluation}

  \textbf{Threats to consider:}
  \begin{enumerate}
    \threat{PHYSICAL}{Physical Access} {
	An attacker can physically access the camera. This can result in destruction of camera in pure vandalizing, removal of camera from premises opening up for other threats (T.X and T.Y), 
	covering camera with other physical object or smearing something on camera housing/lens rendering recorded video useless. }
    \threat{NETWORK}{Network attack} {
	Since the camera is outfitted with a 4G subsystem, attacks relevant for such a system are relevant here. \\
	Note: Go all the way as teacher's pets and specify that object security should be used?}
    \threat{MISMANAGE}{Incompetent personnel} {
	There is a risk that the persons managing and serving the camera do not have the skills required, which can result in password being leaked, physical components of camera damaged, camera not mounted on wall correctly etc.}
    \threat{DISHONESTY}{Dishonest repair personnel}{ %please help me find better name than DISHONEST...
	The repair personnel may be dishonest and try to access the video files, the keys, may try to inject code or access internal memory of camera.}
    \threat{PERSISTENT}{Persisent presence}{
	If the (remote) management system is flawed an attacker could, besides inserting executable foreign code, access camera through the management interface at their convenience.}

    \item{Avoid bugs in the (remote) management that would allow someone to
      insert executable foreign code.}
    \item{Should not be possible to load unauthorized firmware.}
    \item{Don't allow loading of SRTP keys for an incorrect receiver.}
    \item{Include safeguards concerning removing flash memory and loading it
      with altered configurations or code.}
    \item{Deploy routines that circumvents dishonest repair personnel.}
    \item{Secure JTAG debug interface against misuse. Check article about JTAG interface attacking from first part of course.}
    \item{Mitigate damage due to lost key or camera.}
    \item{From first part of course, attack where microprocessor is removed and is either probed or re-soldered (BGA) in order to access memory. Defense against this dark art? Check article again. Note: perhaps part of 'flash memory' entry above in this list.}
  \end{enumerate}

  \comm{%
    Short security evaluation of the design that explained the used security
    measures and the motivation behind them. Should include a list of all
    threats and security issues that are deemed applicable to the system,
    together with any threats that were deemed a non-issue.

    This section contains a table where the problems are listed on the y-axis
    against the solutions on the x-axis where a 'X' marks with solution
    protects against which threat.
  }

    \todo{2-3 pages + summary}

    
\begin{tabular}{| l | c | c | c | c | c | c | c | c | c | c | c | c | c |}
 \cline{2-14}
 \multicolumn{1}{c|}{}  & \rotatebox{90}{O.TPM\_KEY\_STRG} & \rotatebox{90}{O.TRUSTZONE\_NX} & \rotatebox{90}{O.DECOMM} & \rotatebox{90}{O.ID} & \rotatebox{90}{O.NO\_TAMPER} & \rotatebox{90}{O.PWR\_OUT} & \rotatebox{90}{O.ATTEST} & \rotatebox{90}{O.SECURE\_COMMS} & \rotatebox{90}{O.TWO\_WAY\_PROT} & \rotatebox{90}{O.ENC\_DATA} & \rotatebox{90}{O.TPM\_SEAL} & \rotatebox{90}{A.LOCATION} & \rotatebox{90}{A.NO\_ADVERSARIAL} \\
\hline
T.PHYSICAL &   &   &   &   & X &   &   &   &   &   &   & X &   \\
\hline
T.NETWORK &   &   &   &   &   &   &   & X & X &   &   &   &   \\
\hline
T.MISMANAGE &   &   &   &   &   &   &   &   &   &   & X &   & X \\
\hline
T.ADVERSARIAL &   &   &   &   &   &   &   &   &   &   & X &   & X \\
\hline
T.PERSISTENT &   &   &   &   &   &   &   &   &   &   &   &   &   \\
\hline
T.LOST\_ASSET &   &   &   & X &   &   &   &   &   & X & X &   &   \\
\hline
T.MNG\_TEST &   &   &   &   &   &   &   &   &   &   &   &   &   \\
\hline
T.SIGNED\_FW & X &   &   &   &   &   &   &   &   &   &   &   &   \\
\hline
T.SRTP\_RECV &   &   &   &   &   &   &   &   &   &   &   &   &   \\
\hline
T.FLASH\_INTG &   &   &   &   &   &   &   &   &   &   &   &   &   \\
\hline
T.JTAG\_ABUSE &   &   &   &   &   &   &   &   &   & X &   & X &   \\
\hline
\end{tabular}



  \section{Peer reviews}
    \todo{2 pages}

  \section{Improvement sheet}
    \todo{1-2 pages}

  \newpage

  \textbf{Parts of camera:}
  \begin{itemize}
    \item{PCB board.}
    \item{Flash memory.}
    \item{RAM.}
    \item{ARM TrustZone CPU}
    \item{TPM}
    \item{JTAG debug interface.}
    \item{LTE subsystem:}
    \begin{itemize}
      \item{USIM card reader.}
      \item{Specific flash memory to hold ID.}
    \end{itemize}
    \item{Battery holding 3 days operational reserve power.}
  \end{itemize}

  \textbf{Report objects:}
  Come up with a architecture/design that meets following requirements:
  \begin{itemize}
    \item{Security.}
    \item{Production.}
    \item{Maintenance.}
  \end{itemize}
  With motivations!

  \section{General notes}

    \subsection{Acronyms}

    \begin{description}[style=multiline,leftmargin=1.7cm]
      \item[CC]{
          (Common Criteria) Secure systems evaluation criteria adopted by
          $\sim$25 countries.
      }
      \item[TOE]{
          (Target Of Evaluation) The system submitted for evaluation.
      }
      \item[ST]{
          (Security Target) The set of security requirements used as the basis
          for a security evaluations.
      }
      \item[EAL]{
          (Evaluation Assurance Level) The evaluation level that is being
          targeted (there are different levels that can be reached).
      }
      \item[TSF]{
          (TOE Security Functions) The combination of software and hardware
          that is necessary to enforce the chosen policy.
      }
      \item[PP]{
          (Protection Profiles) Type of Common Criteria evaluation. Consists of
          implementation-independent security requirements.
      }
      \item[SFR]{
          (Security Functional Requirements) Describes a functional requirement
          (input $\rightarrow$ behaviour $\rightarrow$ output) related to
          security.
      }
    \end{description}

    \subsection{CC evaluation types}

      There are two types of Common Criteria evaluations.

      \subsubsection{Protection profile evaluation}

        The evaluation is concerned with a set of implementation-independent
        requirements that are established for a category of similar products
        and systems. A typical protection profile includes:
        \begin{itemize}
          \item{Introduction containing a description and overview of the
            targeted system.}
          \item{Description of the products or systems in question.}
          \item{Description of the security environment in relation to the
            products or systems.}
          \item{Relevant security objectives.}
          \item{Requirements on IT security.}
          \item{Rationale behind the grouping.}
        \end{itemize}
        Example of protection profile groupings are; antivirus on workstations,
        biometrics, firewalls and intrusion detection systems.

      \subsubsection{Evaluation against a security target}
      Sources:
      \begin{itemize}
        \item{\url{https://en.wikipedia.org/wiki/Security_Target}}
        \item{\e{slides_on_cc.pdf}}
      \end{itemize}

      \comm{
        \textbf{EAL1: Functionally Tested}

        EAL1 is applicable where some confidence in correct operation is
        required, but the threats to security are not viewed as serious. It
        will be of value where independent assurance is required to support the
        contention that due care has been exercised with respect to the
        protection of personal or similar information. EAL1 provides an
        evaluation of the TOE (Target of Evaluation) as made available to the
        customer, including independent testing against a specification, and an
        examination of the guidance documentation provided. It is intended that
        an EAL1 evaluation could be successfully conducted without assistance
        from the developer of the TOE, and for minimal cost. An evaluation at
        this level should provide evidence that the TOE functions in a manner
        consistent with its documentation, and that it provides useful
        protection against identified threats.

        \textbf{EAL2: Structurally Tested}

        EAL2 requires the cooperation of the developer in terms of the delivery
        of design information and test results, but should not demand more
        effort on the part of the developer than is consistent with good
        commercial practice. As such it should not require a substantially
        increased investment of cost or time.  EAL2 is therefore applicable in
        those circumstances where developers or users require a low to moderate
        level of independently assured security in the absence of ready
        availability of the complete development record. Such a situation may
        arise when securing legacy systems.
      }

        The security target contains a list of implementation-specific security
        requirements that should be met by the system or product in question
        together with the steps taken to meet those requirements. The document
        can also reference relevant protection profiles that are fulfilled.
         A typical security target includes:

        \begin{itemize}
          \item{Introduction of what the target does, key features and purpose.}
          \item{Description of threats and assumptions present in the
            operational environment.}
          \item{Security objectives specifying solutions to the perceived
            threats in both the target and its environment together with the
            rationale these were chosen.}
          \item{Definition of functional and assurance security requirements
            together with the reason these are sufficient and necessary.}
          \item{Target summary specification that enables readers to get a
            general understanding of how the target is structured and
            implemented. This section also provides a high-level view of how
            each security functional requirement is solved through developed
            software.}
        \end{itemize}

  \section{Link dump}
    SRTP protocol definition -- \url{https://tools.ietf.org/html/rfc3711} \\

  \section{Source-code example.}
    \lstset{escapechar=@,style=customc}
    \begin{lstlisting}
      #include <stdio.h>
      #define N 10
      /* Block
       * comment */

      int main()
      {
          int i;

          // Line comment.
          puts("Hello world!");

          for (i = 0; i < N; i++)
          {
              puts("LaTeX is also great for programmers!");
          }

          return 0;
      }
    \end{lstlisting}

\end{document}
