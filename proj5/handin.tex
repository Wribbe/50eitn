\documentclass[10pt]{article}
\usepackage{enumitem}
\usepackage{parskip}
\usepackage[T1]{fontenc}
\usepackage[utf8]{inputenc}
\usepackage{listings}
\usepackage{tikz}
\usepackage{enumitem}
\usepackage[Q=yes]{examplep}
\usepackage{graphicx}
\usepackage{multicol}
\usepackage[hidelinks]{hyperref}
\usepackage{color}
\usepackage{wasysym}
\usepackage{listings}

\newcommand{\command}[1]{\texttt{#1}}
\newcommand{\e}[1]{\PVerb{#1}}
\newcommand{\comm}[1]{{\leavevmode\color{gray}#1}}
\newcommand{\todo}[1]{
  \begin{center}
    [\textcolor{red}{\textbf{\textit{#1}}}]
  \end{center}
}
\newcommand{\threat}[3]{\item{\textbf{T.#1} \hfill \textbf{#2} \\ #3}}
\newcommand{\objective}[3]{\item{\textbf{O.#1} \hfill \textbf{#2} \\ #3}}
\newcommand{\assumption}[3]{\item{\textbf{A.#1} \hfill \textbf{#2} \\ #3}} %These could absolutely be done better, but these will do til we've decided on a format
\newcommand{\objenv}[3]{\item{\textbf{OE.#1} \hfill \textbf{#2} \\ #3}}
\newcommand{\escape}[1]{\PVerb{#1}}

% Check list evnironment.
\newenvironment{checklist}{%
  \begin{list}{}{}%
  \let\olditem\item
  \renewcommand\item{\olditem -- \marginpar{$\Box$} }
  \newcommand\checkeditem{\olditem -- \marginpar{$\CheckedBox$} }
}{%
  \end{list}
}

\lstdefinestyle{customc}{
  belowcaptionskip=1\baselineskip,
  breaklines=true,
  frame=L,
  xleftmargin=\parindent,
  language=C,
  showstringspaces=false,
  basicstyle=\footnotesize\ttfamily,
  keywordstyle=\bfseries\color{green!40!black},
  commentstyle=\itshape\color{purple!40!black},
  identifierstyle=\color{blue},
  stringstyle=\color{orange},
}


\begin{document}

  % Title page.
  %------------

  \thispagestyle{empty}
  \vspace*{3cm}
  \begin{center}
    \huge{EITN50 -- Advanced Computer Security} \\
    \vspace{0.3cm}
    \LARGE{Trusted Camera} \\
    \vspace{1cm}
    \large{by: \\ \vspace{0.2cm}
	\textit{adsec03} \\
        Stefan Eng \texttt{<atn08sen@student.lu.se>} \\
        Rasmus Olofzon \texttt{<muh11rol@student.lu.se>}
        } \\
  \end{center}

  % First page
  %-----------

  \newpage

  \section{Introduction}

    \subsection{TOE Description}

      This document describes a design and implementation of a secure network
      camera platform for video surveillance and monitoring.

      \subsubsection{Intended usage}

        The product is intended to be used as a perimeter security device for
        property owned by a company or the homes of private customers. The
        product is not intended to be used in high security environments such
        as banks or military installations.

      \subsubsection{Security features}

        % TODO: Expand with more security features.
        % TODO: Determine the X in X-bt encrypted RSTP stream.
        The product features secure video streaming capabilities through a
        X-bit encrypted RSTP stream with fully encrypted data storage. Internal
        keys are stored and protected by the onboard TPM.

    \subsubsection{Security assumptions}

       \begin{itemize}
         \assumption{LOCATION}{Mounting place}{The camera is designed to be mounted on outside of building of a tech company.}
         \assumption{CORRECT\_CONFIG}{Correct configuration}{It is assumed that SRTP is used and that authentication and encryption in it are enabled.}
         \assumption{TIMELY\_MAINT}{Maintenance schedule followed}{It is assumed that maintenance personnel follows the maintenance schedule
         and that no deterioration of product due to neglected maintenance occurs.}
         \assumption{QA}{Quality Assurance}{No more than 0.5\% of units are shipped with faulty components.}
       \end{itemize}

  \section{Security problem definition}

    \subsection{Assets}

      Configuration data and stored video recordings.

    \subsection{Threat agents}

      The following attack vectors are deemed relevant for the product.
      Physical attacks in order to disable and dismantle product by
      unauthorized individual inside perimeter. Extracting non-physical
      assets through ill-intentioned maintenance personell, analyzing data in
      traffic or by unauthorized network access.

    \subsection{Threats}
      \begin{itemize}[label={}]
        \threat{PHYSICAL}{Physical Access} {
          An attacker can physically access the camera. This can result in
          destruction of camera in pure vandalizing, removal of camera from
          premises opening up for other threats (T.X and T.Y), covering camera
          with other physical object or smearing something on camera
          housing/lens rendering recorded video useless.
        }
        \threat{NETWORK}{Network attack} {
          Since the camera is outfitted with a 4G subsystem, attacks relevant for
          such a system are relevant here.
          \\ Note: Go all the way as teacher's pets and specify that object
          security should be used?
        }
        \threat{MISMANAGE}{Incompetent personnel} {
          There is a risk that the persons managing and servicing the camera do
          not have the skills required, which can result in password being
          leaked, physical components of camera damaged, camera not mounted on
          wall correctly etc.
        }
        \threat{DISHONESTY}{Dishonest repair personnel}{ %please help me find better name than DISHONEST...
          The repair personnel may be dishonest and try to access the video
          files, the keys, may try to inject code or access internal memory of
          camera.
        }
        \threat{PERSISTENT}{Persistent presence}{
          If the (remote) management system is flawed an attacker could,
          besides inserting executable foreign code, access camera through the
          management interface at their convenience.
        }
        \threat{LOST\_ASSET}{Camera/key is lost}{
          An attacker either physically makes off with a camera or gets hold of
          a crypto key.
        }
        \item{
          Avoid bugs in the (remote) management that would allow someone to
          insert executable foreign code. This should be under T.NETWORK.
        }
        \item{
          Should not be possible to load unauthorized firmware.
        }
        \item{
          Don't allow loading of SRTP keys for an incorrect receiver.
        }
        \item{
          Include safeguards concerning removing flash memory and loading it
          with altered configurations or code.
        }
        \item{
          Secure JTAG debug interface against misuse. Check article about
          JTAG interface attacking from first part of course.
        }
        \item{
          From first part of course, attack where microprocessor is removed
          and is either probed or re-soldered (BGA) in order to access memory.
          Defense against this dark art? Check article again. Note: perhaps
          part of 'flash memory' entry above in this list. Or T.LOST\_ASSET.
        }
      \end{itemize}

    \subsection{Security solutions}

      \subsubsection{Rationale}

      \subsubsection{Coverage}
        
\begin{tabular}{| l | c | c | c | c | c | c | c | c | c | c | c | c | c |}
 \cline{2-14}
 \multicolumn{1}{c|}{}  & \rotatebox{90}{O.TPM\_KEY\_STRG} & \rotatebox{90}{O.TRUSTZONE\_NX} & \rotatebox{90}{O.DECOMM} & \rotatebox{90}{O.ID} & \rotatebox{90}{O.NO\_TAMPER} & \rotatebox{90}{O.PWR\_OUT} & \rotatebox{90}{O.ATTEST} & \rotatebox{90}{O.SECURE\_COMMS} & \rotatebox{90}{O.TWO\_WAY\_PROT} & \rotatebox{90}{O.ENC\_DATA} & \rotatebox{90}{O.TPM\_SEAL} & \rotatebox{90}{A.LOCATION} & \rotatebox{90}{A.NO\_ADVERSARIAL} \\
\hline
T.PHYSICAL &   &   &   &   & X &   &   &   &   &   &   & X &   \\
\hline
T.NETWORK &   &   &   &   &   &   &   & X & X &   &   &   &   \\
\hline
T.MISMANAGE &   &   &   &   &   &   &   &   &   &   & X &   & X \\
\hline
T.ADVERSARIAL &   &   &   &   &   &   &   &   &   &   & X &   & X \\
\hline
T.PERSISTENT &   &   &   &   &   &   &   &   &   &   &   &   &   \\
\hline
T.LOST\_ASSET &   &   &   & X &   &   &   &   &   & X & X &   &   \\
\hline
T.MNG\_TEST &   &   &   &   &   &   &   &   &   &   &   &   &   \\
\hline
T.SIGNED\_FW & X &   &   &   &   &   &   &   &   &   &   &   &   \\
\hline
T.SRTP\_RECV &   &   &   &   &   &   &   &   &   &   &   &   &   \\
\hline
T.FLASH\_INTG &   &   &   &   &   &   &   &   &   &   &   &   &   \\
\hline
T.JTAG\_ABUSE &   &   &   &   &   &   &   &   &   & X &   & X &   \\
\hline
\end{tabular}



  \section{Peer reviews}

  \section{Improvement sheet}

\end{document}
