\documentclass[10pt]{article}
\usepackage{enumitem}
\usepackage{parskip}
\usepackage[T1]{fontenc}
\usepackage[utf8]{inputenc}
\usepackage{listings}
\usepackage{tikz}
\usepackage{enumitem}
\usepackage[Q=yes]{examplep}
\usepackage{graphicx}
\usepackage{multicol}
\usepackage[hidelinks]{hyperref}
\usepackage{color}
\usepackage{wasysym}
\usepackage{listings}
\usepackage{wrapfig}
\usepackage{float}

\newcommand{\command}[1]{\texttt{#1}}
\newcommand{\e}[1]{\PVerb{#1}}
\newcommand{\comm}[1]{{\leavevmode\color{gray}#1}}
\newcommand{\todo}[1]{
  \begin{center}
    [\textcolor{red}{\textbf{\textit{#1}}}]
  \end{center}
}
\newcommand{\threat}[3]{\item{\textbf{T.#1} \hfill \textbf{#2} \\ #3}}
\newcommand{\objective}[3]{\item{\textbf{O.#1} \hfill \textbf{#2} \\ #3}}
\newcommand{\assumption}[3]{\item{\textbf{A.#1} \hfill \textbf{#2} \\ #3}} %These could absolutely be done better, but these will do til we've decided on a format
\newcommand{\objenv}[3]{\item{\textbf{OE.#1} \hfill \textbf{#2} \\ #3}}
\newcommand{\escape}[1]{\PVerb{#1}}

% Check list evnironment.
\newenvironment{checklist}{%
  \begin{list}{}{}%
  \let\olditem\item
  \renewcommand\item{\olditem -- \marginpar{$\Box$} }
  \newcommand\checkeditem{\olditem -- \marginpar{$\CheckedBox$} }
}{%
  \end{list}
}

\lstdefinestyle{customc}{
  belowcaptionskip=1\baselineskip,
  breaklines=true,
  frame=L,
  xleftmargin=\parindent,
  language=C,
  showstringspaces=false,
  basicstyle=\footnotesize\ttfamily,
  keywordstyle=\bfseries\color{green!40!black},
  commentstyle=\itshape\color{purple!40!black},
  identifierstyle=\color{blue},
  stringstyle=\color{orange},
}


\begin{document}

  % Title page.
  %------------

  \thispagestyle{empty}
  \vspace*{3cm}
  \begin{center}
    \huge{EITN50 -- Advanced Computer Security} \\
    \vspace{0.3cm}
    \LARGE{Trusted Camera} \\
    \vspace{1cm}
    \large{by: \\ \vspace{0.2cm}
	\textit{adsec03} \\
        Stefan Eng \texttt{<atn08sen@student.lu.se>} \\
        Rasmus Olofzon \texttt{<muh11rol@student.lu.se>}
        } \\
  \end{center}

  % First page
  %-----------

  \newpage

  \section{Introduction}

    %Short high-level architectural overview (guessing hardware? Think both hardware and software) that
    %describes how the design is structured. Motivate all design-choices,
    %don't go into too much detail, the level of the description is the
    %important part. The design should be presented in such a way that it is
    %easy to understand. This section also includes a half page illustration
    %of the structure together with the Target Of Evaluation (TOE) and
    %Security Target (ST).

    %The main point is to exercise technical writing ability. Said description
    %should only include details that are important from a security
    %perspective. The intended audience is a student that has completed the
    %basic computer security course.

    \subsection{TOE Description}

      This document describes the design and implementation of a secure network
      camera platform for video surveillance and monitoring.

      \begin{figure}[!h]
        \center
        \includegraphics[width=\textwidth]{input/pcb_camera.png}
      \end{figure}

      \subsubsection{Intended usage}

        The product is intended to be used as a perimeter security device for
        property owned by a company or the homes of private customers. The
        product is not intended to be used in high security environments such
        as banks or military installations.

      \subsubsection{Considered design requirements}

        % TODO: Write about all the requirements from the introduction.

      \subsubsection{Security features}

        % TODO: Expand with more security features.
        % TODO: Determine the X in X-bt encrypted RSTP stream.
        % TODO: Add tamper protection.
        The product features secure video streaming capabilities through a
        X-bit encrypted RSTP stream with fully encrypted data storage. Internal
        keys are stored and protected by the onboard TPM.

      \subsubsection{Security assumptions}

         \begin{itemize}
           \assumption{LOCATION}{Mounting place}{
             The product is assumed to be mounted on the outer or inner walls of
             a building at a height that requires a small ladder to reach it.
           }
           \assumption{CORRECT\_CONFIG}{Correct configuration}{
             The running product is assumed to have a correct configuration that
             enables a fully encrypted and authenticated SRTP stream to be used
             for data transfer.
           }
           \assumption{TIMELY\_MAINT}{Maintenance schedule followed}{
             It is assumed that the maintenance schedule is followed and that any
             defects or deteriorations outside of regular maintenance are
             reported as soon as they are detected.
           }
           % TODO: Ha kvar denna?
           \assumption{QA}{Quality Assurance}{
             No more than 0.5\% of units are shipped with faulty components.
           }
         \end{itemize}

  % TODO: Motsvarar detta ST?
  \section{Security problem definition}

    \subsection{Assets}

      The product contains two types of valuable assets; physical and
      non-physical. Among the product hardware, the TPM and flash memory
      are of special interest from a security perspective since these house
      configuration data, stored video recordings and security related
      information. The non-physical assets consist of security keys used for
      encrypting communication, video-transmission and storage, together with
      the video recordings themselves and possible sensitive configuration
      information.

    \subsection{Threat agents}

      The following attack vectors are deemed relevant for the product.
      Physical attacks in order to disable and dismantle product by
      unauthorized individual inside perimeter. Extracting non-physical
      assets ill-intentioned maintenance personell, analyzing data in
      traffic or by unauthorized network access.

    \subsection{Threats}
      \begin{itemize}[label={}]
        \threat{PHYSICAL}{Physical Access} {
          An attacker can physically access the camera. This can result in
          destruction of camera in pure vandalizing, removal of camera from
          premises opening up for other threats (T.X and T.Y), covering camera
          with other physical object or smearing something on camera
          housing/lens rendering recorded video useless.
        }
        \threat{NETWORK}{Network attack} {
          Since the camera is outfitted with a 4G subsystem, attacks relevant for
          such a system are relevant here.
          \\ Note: Go all the way as teacher's pets and specify that object
          security should be used?
        }
        \threat{MISMANAGE}{Incompetent personnel} {
          There is a risk that the persons managing and servicing the camera do
          not have the skills required, which can result in password being
          leaked, physical components of camera damaged, camera not mounted on
          wall correctly etc.
        }
        \threat{Adversarial}{Adversarial repair personnel}{ %better?
          The repair personnel may be dishonest and try to access the video
          files, the keys, may try to inject code or access internal memory of
          camera.
        }
        \threat{PERSISTENT}{Persistent presence}{
          If the (remote) management system is flawed an attacker could,
          besides inserting executable foreign code, access camera through the
          management interface at their convenience.
        }
        \threat{LOST\_ASSET}{Camera/key is lost}{
          An attacker either physically makes off with a camera or gets hold of
          a crypto key.
        }
        \threat{MANAGEMENT\_TESTING}{Faulty management software}{
          Avoid bugs in the (remote) management that would allow someone to
          insert executable foreign code. This should be under T.NETWORK.
        }
        \threat{SIGNED\_FIRMWARE}{Running unauthorized firmware}{
          Should not be possible to load unauthorized firmware.
        }
        \threat{SRTP\_RECV}{Loading non-matching SRTP keys}{
          Don't allow loading of SRTP keys for an incorrect receiver.
        }
        \threat{FLASHMEM\_INTEGRITY}{Flashing faulty code/config}{
          Include safeguards concerning removing flash memory and loading it
          with altered configurations or code.
        }
        \threat{JTAG\_ABUSE}{Unauthorized access through JTAG}{
          Secure JTAG debug interface against misuse. Check article about
          JTAG interface attacking from first part of course.
        }

        % TODO: Do we want this?
        \threat{DESOLDER}{Dismantling/modifying physical components}{
          From first part of course, attack where microprocessor is removed
          and is either probed or re-soldered (BGA) in order to access memory.
          Defense against this dark art? Check article again. Note: perhaps
          part of 'flash memory' entry above in this list. Or T.LOST\_ASSET.
        }
      \end{itemize}

    \subsection{Security solutions}

      %Short security evaluation of the design that explained the used security
      %measures and the motivation behind them. Should include a list of all
      %threats and security issues that are deemed applicable to the system,
      %together with any threats that were deemed a non-issue.

      %This section contains a table where the problems are listed on the y-axis
      %against the solutions on the x-axis where a 'X' marks with solution
      %protects against which threat.

      \subsubsection{Rationale}

      \subsubsection{Coverage}
        
\begin{tabular}{| r | c | c | c | c | c | c | c | c | c |}
 \cline{2-10}
 \multicolumn{1}{c|}{}  & \rotatebox{90}{O.TPM\_KEY\_STRG} & \rotatebox{90}{O.TRUSTZONE\_NX} & \rotatebox{90}{O.DECOMM} & \rotatebox{90}{O.ID} & \rotatebox{90}{O.NO\_TAMPER} & \rotatebox{90}{O.PWR\_OUT} & \rotatebox{90}{O.ATTEST} & \rotatebox{90}{O.SECURE\_COMMS} & \rotatebox{90}{O.TWO\_WAY\_PROT} \\
\hline
T.PHYSICAL & X & X & X &   &   &   &   &   &   \\
\hline
T.NETWORK &   & X &   &   &   &   &   &   &   \\
\hline
T.MISMANAGE &   &   &   & X & X &   &   &   &   \\
\hline
T.ADVESERIAL &   &   &   &   &   & X & X &   &   \\
\hline
T.PERSISTENT &   &   &   &   &   & X &   &   & X \\
\hline
T.LOST\_ASSET &   &   &   &   & X &   &   & X &   \\
\hline
T.MANAGEMENT\_TESTING &   & X & X &   &   &   &   &   &   \\
\hline
T.SIGNED\_FIRMWARE & X & X & X & X & X &   &   &   &   \\
\hline
T.SRTP\_RECV &   &   & X &   & X & X &   &   &   \\
\hline
T.FLASHMEM\_INTEGRITY &   & X &   & X &   &   &   &   &   \\
\hline
T.JTAG\_ABUSE & X &   &   &   &   &   &   & X &   \\
\hline
\end{tabular}



  \section{Peer reviews}

  \section{Improvement sheet}

\end{document}
