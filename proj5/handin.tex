\documentclass[10pt]{article}
\usepackage{enumitem}
\usepackage{parskip}
\usepackage[T1]{fontenc}
\usepackage[utf8]{inputenc}
\usepackage{listings}
\usepackage{tikz}
\usepackage{enumitem}
\usepackage[Q=yes]{examplep}
\usepackage{graphicx}
\usepackage{multicol}
\usepackage[hidelinks]{hyperref}
\usepackage{color}
\usepackage{wasysym}
\usepackage{listings}
\usepackage{wrapfig}
\usepackage{float}

\newcommand{\command}[1]{\texttt{#1}}
\newcommand{\e}[1]{\PVerb{#1}}
\newcommand{\comm}[1]{{\leavevmode\color{gray}#1}}
\newcommand{\todo}[1]{
  \begin{center}
    [\textcolor{red}{\textbf{\textit{#1}}}]
  \end{center}
}
\newcommand{\threat}[3]{\item{\textbf{T.#1} \hfill \textbf{#2} \\ #3}}
\newcommand{\objective}[3]{\item{\textbf{O.#1} \hfill \textbf{#2} \\ #3}}
\newcommand{\assumption}[3]{\item{\textbf{A.#1} \hfill \textbf{#2} \\ #3}} %These could absolutely be done better, but these will do til we've decided on a format
\newcommand{\objenv}[3]{\item{\textbf{OE.#1} \hfill \textbf{#2} \\ #3}}
\newcommand{\escape}[1]{\PVerb{#1}}

% Check list evnironment.
\newenvironment{checklist}{%
  \begin{list}{}{}%
  \let\olditem\item
  \renewcommand\item{\olditem -- \marginpar{$\Box$} }
  \newcommand\checkeditem{\olditem -- \marginpar{$\CheckedBox$} }
}{%
  \end{list}
}

\lstdefinestyle{customc}{
  belowcaptionskip=1\baselineskip,
  breaklines=true,
  frame=L,
  xleftmargin=\parindent,
  language=C,
  showstringspaces=false,
  basicstyle=\footnotesize\ttfamily,
  keywordstyle=\bfseries\color{green!40!black},
  commentstyle=\itshape\color{purple!40!black},
  identifierstyle=\color{blue},
  stringstyle=\color{orange},
}


\begin{document}

  % Title page.
  %------------

  \thispagestyle{empty}
  \vspace*{3cm}
  \begin{center}
    \huge{EITN50 -- Advanced Computer Security} \\
    \vspace{0.3cm}
    \LARGE{Trusted Camera} \\
    \vspace{1cm}
    \large{by: \\ \vspace{0.2cm}
	\textit{adsec03} \\
        Stefan Eng \texttt{<atn08sen@student.lu.se>} \\
        Rasmus Olofzon \texttt{<muh11rol@student.lu.se>}
        } \\
  \end{center}

  % First page
  %-----------

  \newpage

  \section{Introduction}

    %Short high-level architectural overview (guessing hardware? Think both hardware and software) that
    %describes how the design is structured. Motivate all design-choices,
    %don't go into too much detail, the level of the description is the
    %important part. The design should be presented in such a way that it is
    %easy to understand. This section also includes a half page illustration
    %of the structure together with the Target Of Evaluation (TOE) and
    %Security Target (ST).

    %The main point is to exercise technical writing ability. Said description
    %should only include details that are important from a security
    %perspective. The intended audience is a student that has completed the
    %basic computer security course.

    \subsection{TOE Description}

      This document describes the design and implementation of a secure network
      camera platform for video surveillance and monitoring.

      \begin{figure}[!h]
        \center
        \includegraphics[width=\textwidth]{input/pcb_camera.png}
      \end{figure}

      \begin{figure}[!h]
        \center
        \includegraphics[width=0.7\textwidth]{input/communication.pdf}
      \end{figure}

      \subsubsection{Intended usage}

        The product is intended to be used as a perimeter security device for
        property owned by a company or the homes of private customers. The
        product is not intended to be used in high security environments such
        as banks or military installations.

      \subsubsection{Considered design requirements}

        % TODO: Write about all the requirements from the introduction.

      \subsubsection{Security assumptions}

         \begin{itemize}
           \assumption{LOCATION}{Mounting place}{
             The product is assumed to be mounted on the outer or inner walls of
             a building at a height that requires a small ladder to reach it.
           }
           \assumption{TIMELY\_MAINT}{Maintenance schedule followed}{
             It is assumed that the maintenance schedule is followed and that any
             defects or deteriorations outside of regular maintenance are
             reported as soon as they are detected.
           }
         \end{itemize}

  \section{Security architecture -- overview}

    \begin{figure}[!h]
      \center
      \includegraphics[width=\textwidth]{input/security_layers.pdf}
      \caption{Schematic representation of hardware and software security layers.}
    \end{figure}

    \subsection{Hardware security layer}

      The hardware security layer of the platform consists of a ARM TrustZone
      CPU, a Trusted Platform Module for managing software security related
      artefacts and a static unique identifier created during the platform
      production.

      \subsubsection{ARM TrustZone CPU}

        The ARM TrustZone CPU provides hardware level separation of trusted and
        untrusted region in the computational space. Using this feature in
        combination with a trusted OS and trusted boot a Trusted Execution
        Environment is created which can then act as a host to any trusted
        applications that will run on the platform.

        \begin{figure}[!h]
          \center
          % TODO: Make a draw.io copy of this picture.
          \includegraphics[width=0.5\textwidth]{input/arm_trust.jpg}
          \caption{ARM TrustZone security separation.}
        \end{figure}

        The CPU also features W xor X-protection by enabling its Nx-bit.
        Activating this feature reduces the attack-surface for code-injection
        attempts by making user-space memory either executable or modifiable,
        but not both. Since the platform includes management software for
        extracting operational information and remote configuration, this
        feature is of great interest.


      \subsubsection{Trusted Platform Module}

      \subsubsection{Unique ID}

      \subsubsection{Tamper protection}

    \subsection{Software security layer}

      \subsubsection{Authenticated user}
      \subsubsection{Encrypted data streams}
      \subsubsection{Storage encryption}
      \subsubsection{Verified firmware}

    \begin{itemize}
      \item User, connected to: '4G modem'
      \begin{itemize}
        \item TPM
      \end{itemize}
      \item Maintenance Person, connected to: \{'4G modem' for remote management, 'JTAG debug port' for close management\}
      \begin{itemize}
        \item TPM
      \end{itemize}
      \item Power supply, connected to: 'Battery'
      \item Repair Person, connected to: all components of camera %state this, don't draw it..
    \end{itemize}
    \textbf{Changes to PCB picture}
    \begin{itemize}
      \item Change 'button/sensor1' to 'Tamper switch'
      \item Remove 'button/sensor2'
      \item Add 'Burnt-in Master Key', connected to: TPM(?) %R@S: where is this 'burnt in'?
      \item Add text 'SRTP', connected to: 4G modem antenna
      \item ? Add text 'Encryption of video data', connected to path between CPU and Flash memory ? % perhaps too detailed
      %\item Add `Key Blobs (TPM)', in Flash
    \end{itemize}
    \textbf{TPM}
    \begin{itemize} %perhaps not have keys at all
      \item EK
      \item SRK
      \item AIK
    \end{itemize}

    \subsubsection{Life cycle of TOE}
    \begin{itemize}
      \item 'Production', connected to: 'User operation'
      \item 'User operation', conn to: \{'Service/repair', 'Maintenance'\}
      \item 'Maintenance', conn to: 'User operation'
      \item 'Service/repair', conn to: \{'User operation', 'Decommission'\}
      \item 'Decommission'
    \end{itemize}

	\subsubsection{Assets}

      The product contains two types of valuable assets; physical and
      non-physical. Among the product hardware, the TPM and flash memory
      are of special interest from a security perspective since these house
      configuration data, stored video recordings and security related
      information. The non-physical assets consist of security keys used for
      encrypting communication, video-transmission and storage, together with
      the video recordings themselves and possible sensitive configuration
      information.

 \subsubsection{Security objectives for the TOE}
	\begin{itemize}
		\objective{TPM\_KEY\_STRG}{Storage of keys in camera}{
      The storage of internal keys of camera should be done in the TPM.
    }
		\objective{TRUSTZONE\_NX}{Nx enabled in ARM CPU}{
      The ARM Trustzone CPU should have Nx bit enabled.
    }
    % (W xor X-protection from project 4, to protect against code injection).
    % Could perhaps add more on trustzone, from Lect6, security rings and
    % running code in privileged mode etc. Could perhaps mitigate some threats
    % here. Last slide of Lect6 says embedded cpus work differently though, and
    % here we use an embedded system, I would say.
		\objective{DECOMM}{Easy erasure of data}{
      When the TOE is decommissioned, it should be easy for the user to make
      stored user information and data stored unreadable (or wiped).
    }
		\objective{ID}{Strong identity}{
      The camera should have a unique, cryptographically strong, identity that
      is programmed during camera manufacturing.
    }
		\objective{NO\_TAMPER}{Detect tampering}{
      The TOE should have a tamper switch on backside of housing, in order to
      detect if pulled off of wall.
    }
		\objective{PWR\_OUT}{Notify if lost power}{
      A notification should be sent if connection to power source is cut off.
    }
		\objective{ATTEST}{Attestation used for remote}{
      When a Maintenance Person is to update firmware, attestation should be
      carried out with the help of the TPMs in MP's computer and the TOE.
    }
		 \objective{SECURE\_COMMS}{Secure communictions}{
             The TOE should have a correct configuration that
             enables a fully encrypted and authenticated SRTP stream to be used
             for data transfer, and also use it.
           }
	\end{itemize}
        % TODO: Expand with more security features.
        % TODO: Determine the X in X-bt encrypted RSTP stream.
        % TODO: Add tamper protection.
        The product features secure video streaming capabilities through a
        X-bit encrypted SRTP stream with fully encrypted data storage.

        % Split it into hardware/software?

        Forwards and backwards protection by reapplying the key derivation
        function for the SRTP connection.

  \subsubsection{Security Objectives for the Operational Environment}
	\begin{itemize}
		\objenv{NO\_ADVERSARIAL}{No adversarial personnel}{
      It is assumed that background checks on repair personnel are carried out.
    } % R: maybe not have this, but could mitigate T_ADVERSARIAL. Otherwise, an
      % example of OpEnv-obj
	\end{itemize}

  % TODO: Motsvarar detta ST? R: Nej, jag tror att i ST:n ingår threats och
  % objectives för TOE:n. Flyttar därför upp. Edit: Bör göra det utifrån
  % exempel jag sett, men i projektbeskrivningen står uttryckligen att
  % threat-listan ska vara i andra delen. Flyttade därför ner den igen...
  % Edit2: Wikipedia säger också att threats bör vara ovan..
  % https://en.wikipedia.org/wiki/Security_Target

  \section{Security problem definition}

    \subsection{Security solutions}

      %Short security evaluation of the design that explained the used security
      %measures and the motivation behind them. Should include a list of all
      %threats and security issues that are deemed applicable to the system,
      %together with any threats that were deemed a non-issue.

      %This section contains a table where the problems are listed on the y-axis
      %against the solutions on the x-axis where a 'X' marks with solution
      %protects against which threat.

 \subsubsection{Threat agents}

      The following attack vectors are deemed relevant for the product.
      Physical attacks in order to disable and dismantle product by
      unauthorized individual inside perimeter. Extracting non-physical
      assets ill-intentioned maintenance personnel, analyzing data in
      traffic or by unauthorized network access.

    \subsubsection{Threats}
      \begin{itemize}[label={}]
        \threat{PHYSICAL}{Physical Access} {
          An attacker can physically access the camera. This can result in
          destruction of camera in pure vandalizing, removal of camera from
          premises opening up for other threats (T.FLASHMEM\_INTEGRITY and T.LOST\_ASSET), covering camera
          with other physical object or smearing something on camera
          housing/lens rendering recorded video useless.
        }
        \threat{NETWORK}{Network attack} {
          Since the camera is outfitted with a 4G subsystem, attacks relevant for
          such a system are relevant here. }
        \threat{MISMANAGE}{Incompetent personnel} {
          There is a risk that the persons managing and servicing the camera do
          not have the skills required, which can result in password being
          leaked, physical components of camera damaged, camera not mounted on
          wall correctly etc.
        }
        \threat{ADVERSARIAL}{Adversarial repair personnel}{ %better? R: very fancy
          The repair personnel may be dishonest and try to access the video
          files, the keys, may try to inject code or access internal memory of
          camera.
        }
        \threat{PERSISTENT}{Persistent presence}{
          If the (remote) management system is flawed an attacker could, access camera through the
          management interface at their convenience.
        }
        \threat{LOST\_ASSET}{Camera/key is lost}{
          An attacker either physically makes off with a camera or gets hold of
          a crypto key.
        }
        \threat{MANAGEMENT\_TESTING}{Faulty management software}{
          If bugs exist in the (remote) management system, that would allow an attacker to
          insert executable foreign code.
        }
        \threat{SIGNED\_FIRMWARE}{Running unauthorized firmware}{
          An attacker could load unauthorized firmware.
        }
        \threat{SRTP\_RECV}{Loading non-matching SRTP keys}{
          An attacker could load SRTP keys for an incorrect receiver.
        }
        \threat{FLASHMEM\_INTEGRITY}{Flashing faulty code/config}{
          Include safeguards concerning removing flash memory and loading it
          with altered configurations or code.
        }
        \threat{JTAG\_ABUSE}{Unauthorized access through JTAG}{
          Secure JTAG debug interface against misuse. %Check article about JTAG interface attacking from first part of course.
        }

        % TODO: Do we want this?
        %\threat{DESOLDER}{Dismantling/modifying physical components}{
          %From first part of course, attack where microprocessor is removed
          %and is either probed or re-soldered (BGA) in order to access memory.
          %Defense against this dark art? Check article again. Note: perhaps
          %part of 'flash memory' entry above in this list. Or T.LOST\_ASSET.
        %}
      \end{itemize}

      \subsubsection{Rationale}
	%Here, could do tables as in https://www.commoncriteriaportal.org/files/epfiles/ST-FFHDD.pdf

      \subsubsection{Coverage}
        
\begin{tabular}{| r | c | c | c | c | c | c | c | c | c |}
 \cline{2-10}
 \multicolumn{1}{c|}{}  & \rotatebox{90}{O.TPM\_KEY\_STRG} & \rotatebox{90}{O.TRUSTZONE\_NX} & \rotatebox{90}{O.DECOMM} & \rotatebox{90}{O.ID} & \rotatebox{90}{O.NO\_TAMPER} & \rotatebox{90}{O.PWR\_OUT} & \rotatebox{90}{O.ATTEST} & \rotatebox{90}{O.SECURE\_COMMS} & \rotatebox{90}{O.TWO\_WAY\_PROT} \\
\hline
T.PHYSICAL & X & X & X &   &   &   &   &   &   \\
\hline
T.NETWORK &   & X &   &   &   &   &   &   &   \\
\hline
T.MISMANAGE &   &   &   & X & X &   &   &   &   \\
\hline
T.ADVESERIAL &   &   &   &   &   & X & X &   &   \\
\hline
T.PERSISTENT &   &   &   &   &   & X &   &   & X \\
\hline
T.LOST\_ASSET &   &   &   &   & X &   &   & X &   \\
\hline
T.MANAGEMENT\_TESTING &   & X & X &   &   &   &   &   &   \\
\hline
T.SIGNED\_FIRMWARE & X & X & X & X & X &   &   &   &   \\
\hline
T.SRTP\_RECV &   &   & X &   & X & X &   &   &   \\
\hline
T.FLASHMEM\_INTEGRITY &   & X &   & X &   &   &   &   &   \\
\hline
T.JTAG\_ABUSE & X &   &   &   &   &   &   & X &   \\
\hline
\end{tabular}



  \section{Peer reviews}

  \section{Improvement sheet}

\end{document}
